% Новые переменные, которые могут использоваться во всём проекте
% ГОСТ 7.0.11-2011
% 9.2 Оформление текста автореферата диссертации
% 9.2.1 Общая характеристика работы включает в себя следующие основные структурные
% элементы:
% актуальность темы исследования;
\newcommand{\actualityTXT}{Актуальность темы.}
\newcommand{\actualityTXTEn}{Relevance.}
% степень ее разработанности;
\newcommand{\progressTXT}{Степень разработанности темы.}
\newcommand{\progressTXTEn}{State of the field.}
% цели и задачи;
\newcommand{\aimTXT}{Целью}
\newcommand{\aimTXTEn}{The aim}
\newcommand{\tasksTXT}{задачи}
\newcommand{\tasksTXTEn}{tasks}
% научную новизну;
\newcommand{\noveltyTXT}{Научная новизна:}
\newcommand{\noveltyTXTEn}{The scientific novelty:}
% теоретическую и практическую значимость работы;
%\newcommand{\influenceTXT}{Теоретическая и практическая значимость}
% или чаще используют просто
\newcommand{\influenceTXT}{Практическая значимость}
\newcommand{\influenceTXTEn}{The practical significance}
% методологию и методы исследования;
\newcommand{\methodsTXT}{Методология и методы исследования.}
\newcommand{\methodsTXTEn}{Study methodology and methods.}
% положения, выносимые на защиту;
\newcommand{\defpositionsTXT}{Основные положения, выносимые на~защиту:}
\newcommand{\defpositionsTXTEn}{The principal statements of the~thesis:}
% степень достоверности и апробацию результатов.
\newcommand{\reliabilityTXT}{Достоверность}
\newcommand{\reliabilityTXTEn}{The validity and reliability}
\newcommand{\probationTXT}{Апробация работы.}
\newcommand{\probationTXTEn}{Approbation.}

\newcommand{\contributionTXT}{Личный вклад}
\newcommand{\contributionTXTEn}{Personal contribution.}
\newcommand{\implementationTXT}{Внедрение результатов работы.}
\newcommand{\implementationTXTEn}{Application.}
\newcommand{\publicationsTXT}{Публикации.}
\newcommand{\publicationsTXTEn}{Publications.}

%%% Заголовки библиографии:

% для автореферата:
\newcommand{\bibtitleauthor}{Публикации автора по теме диссертации}
\newcommand{\bibtitleauthorEn}{List of own publications}

% для стиля библиографии `\insertbiblioauthorgrouped`
\newcommand{\bibtitleauthorvak}{В изданиях из списка ВАК РФ}
\newcommand{\bibtitleauthorvakEn}{Publications in editions from the list of the Higher Attestation Commission}


\newcommand{\bibtitleauthorscopus}{В изданиях, входящих в международную базы цитирования Scopus и Web of Science}
\newcommand{\bibtitleauthorscopusEn}{Publications in international editions, indexed in Scopus and WoS}

\newcommand{\bibtitleauthorwos}{В изданиях, входящих в международную базу цитирования Web of Science}

\newcommand{\bibtitleauthorother}{В прочих изданиях}
\newcommand{\bibtitleauthorotherEn}{In other editions}

\newcommand{\bibtitleauthorconf}{В сборниках трудов конференций}
\newcommand{\bibtitleauthorpatent}{Зарегистрированные патенты}
\newcommand{\bibtitleauthorprogram}{Зарегистрированные программы для ЭВМ}

% для стиля библиографии `\insertbiblioauthorimportant`:
\newcommand{\bibtitleauthorimportant}{Наиболее значимые \protect\MakeLowercase\bibtitleauthor}

% для списка литературы в диссертации и списка чужих работ в автореферате:
\newcommand{\bibtitlefull}{Список литературы} % (ГОСТ Р 7.0.11-2011, 4)
