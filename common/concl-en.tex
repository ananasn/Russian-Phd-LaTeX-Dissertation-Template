%% Согласно ГОСТ Р 7.0.11-2011:
%% 5.3.3 В заключении диссертации излагают итоги выполненного исследования, рекомендации, перспективы дальнейшей разработки темы.
%% 9.2.3 В заключении автореферата диссертации излагают итоги данного исследования, рекомендации и перспективы дальнейшей разработки темы.

The significance of the results obtained in the dissertation work is emphasized by the possibility of integrating the developed methods, approaches, and algorithms into the computer-aided design system of modular technological equipment. The proposed recommendations for the creation of units and assemblies and control units for modular technological equipment will simplify its development and use in the conditions of single and small-scale production. During the research, the following main theoretical and practical results were obtained:

\begin{enumerate}
\item A universal design of re-adjustable modular equipment is proposed, consisting of a universal coordinate chassis with a movable carriage; modules that determine the main operation performed by the equipment and are installed on the suspension of the chassis carriage using an electromagnetic fastening; and a modular control unit that implements a numerical control algorithm.
\item Algorithmic and software and hardware support for the modular control unit has been developed, which includes a set of software and hardware tools for implementing decentralized network interaction of modules and software for automatic reconfiguration of a unit of modular equipment, which reduces the complexity of the changeover.
\item A method for unifying modules with an electromagnetic mount has been developed, which includes a method for determining the parameters of unification and their limitations and a method for forming a parametric series based on the formulated limitations.
\item A criterion for the expediency of using modular equipment is proposed, based on the analysis of group technological processes, making it possible to assess the prospects for using modular equipment for typical technological processes used at an enterprise.
\item A method for optimizing a set of modular equipment has been developed, which includes a method for calculating the weight coefficients of the optimization objective function and a two-criterion optimization algorithm based on the theory of normalization and the discrete-event method.
\item Based on the methodology for optimizing a set of modular equipment, software was developed, and a numerical experiment was carried out, which showed an increase in the productivity of a group of products by~\SI{18}{\percent}.
\item To test the proposed design of modular equipment and methods and algorithms for working with it, a prototype of a modular technological platform was created.
\end{enumerate}

Thus, based on the results obtained, the goal of this dissertation research on the development of methods for the design and use of modular technological equipment in the conditions of single and small-scale production can be considered achieved.

The results obtained correspond to paragraph 6 ``Development, research and implementation of new types of technological equipment for the manufacture of parts, assembly, adjustment, control and testing of devices'' and paragraph 7 ``development and implementation of new methods and means of mechanization, automation, robotization of instrument-making production, providing an increase in productivity, a decrease in labor intensity and an increase in the efficiency of production'' specialty passport 05.11.14---`` Instrument-making technology''.
