%%% Основные сведения %%%
\newcommand{\thesisAuthorLastName}{Крылова}
\newcommand{\thesisAuthorOtherNames}{Анастасия Андреевна}
\newcommand{\thesisAuthorLastNameEn}{Krylova}
\newcommand{\thesisAuthorOtherNamesEn}{Anastasiya Andreevna}
\newcommand{\thesisAuthorInitials}{А.\,А.}
\newcommand{\thesisAuthorEn}{\thesisAuthorLastNameEn~\thesisAuthorOtherNamesEn}
\newcommand{\thesisAuthor}             % Диссертация, ФИО автора
{%
    \texorpdfstring{% \texorpdfstring takes two arguments and uses the first for (La)TeX and the second for pdf
        \thesisAuthorLastName~\thesisAuthorOtherNames% так будет отображаться на титульном листе или в тексте, где будет использоваться переменная
    }{%
        \thesisAuthorLastName, \thesisAuthorOtherNames% эта запись для свойств pdf-файла. В таком виде, если pdf будет обработан программами для сбора библиографических сведений, будет правильно представлена фамилия.
    }
}
\newcommand{\thesisAuthorShort}        % Диссертация, ФИО автора инициалами
{\thesisAuthorInitials~\thesisAuthorLastName}
%\newcommand{\thesisUdk}                % Диссертация, УДК
%{\fixme{xxx.xxx}}
\newcommand{\thesisTitle}              % Диссертация, название
{Исследование и~разработка модульного технологического оборудования для единичного и~мелкосерийного производства}
\newcommand{\thesisTitleEn}
{Research and Development of Modular Technological Equipment for Job and Small-batch production}
\newcommand{\thesisSpecialtyNumber}    % Диссертация, специальность, номер
{05.11.14}
\newcommand{\thesisSpecialtyTitle}     % Диссертация, специальность, название (название взято с сайта ВАК для примера)
{Технология приборостроения}
\newcommand{\thesisSpecialtyTitleEn}
{Instrumentation technology}
%% \newcommand{\thesisSpecialtyTwoNumber} % Диссертация, вторая специальность, номер
%% {\fixme{XX.XX.XX}}
%% \newcommand{\thesisSpecialtyTwoTitle}  % Диссертация, вторая специальность, название
%% {\fixme{Теория и~методика физического воспитания, спортивной тренировки,
%% оздоровительной и~адаптивной физической культуры}}
\newcommand{\thesisDegree}             % Диссертация, ученая степень
{кандидата технических наук}
\newcommand{\thesisDegreeEn}
{сandidate of engineering}
\newcommand{\thesisDegreeShort}        % Диссертация, ученая степень, краткая запись
{канд. техн. наук}
\newcommand{\thesisCity}               % Диссертация, город написания диссертации
{Санкт-Петербург}
\newcommand{\thesisCityEn}
{Saint-Petersburg}
\newcommand{\thesisYear}               % Диссертация, год написания диссертации
{\the\year}
\newcommand{\thesisOrganization}       % Диссертация, организация
{Национальный исследовательский университет ИТМО\\(Университет ИТМО)}
\newcommand{\thesisOrganizationEn}
{ITMO University}
\newcommand{\thesisOrganizationShort}  % Диссертация, краткое название организации для доклада
{Университет ИТМО}

\newcommand{\thesisInOrganization}     % Диссертация, организация в предложном падеже: Работа выполнена в ...
{федеральном государственном автономном образовательном учреждении высшего образования <<Национальный исследовательский университет ИТМО>>}

%% \newcommand{\supervisorDead}{}           % Рисовать рамку вокруг фамилии
\newcommand{\supervisorFio}              % Научный руководитель, ФИО
{Афанасьев Максим Яковлевич}
\newcommand{\supervisorFioEn}              % Научный руководитель, ФИО
{Afanasev Maksim Ya.}
\newcommand{\supervisorRegalia}          % Научный руководитель, регалии
{кандидат технических наук}
\newcommand{\supervisorRegaliaEn}
{candidate of engineering sciences}
\newcommand{\supervisorFioShort}         % Научный руководитель, ФИО
{М.\,Я.~Афанасьв}
\newcommand{\supervisorRegaliaShort}     % Научный руководитель, регалии
{к.\,т.\,н.}

%% \newcommand{\supervisorTwoDead}{}        % Рисовать рамку вокруг фамилии
%% \newcommand{\supervisorTwoFio}           % Второй научный руководитель, ФИО
%% {\fixme{Фамилия Имя Отчество}}
%% \newcommand{\supervisorTwoRegalia}       % Второй научный руководитель, регалии
%% {\fixme{уч. степень, уч. звание}}
%% \newcommand{\supervisorTwoFioShort}      % Второй научный руководитель, ФИО
%% {\fixme{И.\,О.~Фамилия}}
%% \newcommand{\supervisorTwoRegaliaShort}  % Второй научный руководитель, регалии
%% {\fixme{уч.~ст.,~уч.~зв.}}

\newcommand{\opponentOneFio}           % Оппонент 1, ФИО
{\fixme{Фамилия Имя Отчество}}
\newcommand{\opponentOneFioEn}
{\fixme{Surname Name Patronimic}}
%
\newcommand{\opponentOneRegalia}       % Оппонент 1, регалии
{\fixme{доктор физико-математических наук, профессор}}
\newcommand{\opponentOneRegaliaEn}      
{\fixme{Doctor of Technical Sciences, Professor}}
%
\newcommand{\opponentOneJobPlace}      % Оппонент 1, место работы
{\fixme{Не очень длинное название для места работы}}
\newcommand{\opponentOneJobPlaceEn}      
{\fixme{Place of work}}
%
\newcommand{\opponentOneJobPost}       % Оппонент 1, должность
{\fixme{старший научный сотрудник}}
\newcommand{\opponentOneJobPostEn}
{\fixme{position}}
%
\newcommand{\opponentTwoFio}           % Оппонент 2, ФИО
{\fixme{Фамилия Имя Отчество}}
\newcommand{\opponentTwoFioEn}
{\fixme{Surname Name Patronimic}}
%
\newcommand{\opponentTwoRegalia}       % Оппонент 2, регалии
{\fixme{кандидат технических наук}}
\newcommand{\opponentTwoRegaliaEn}
{\fixme{Doctor of Technical Sciences, Professor}}
%
\newcommand{\opponentTwoJobPlace}      % Оппонент 2, место работы
{\fixme{Основное место работы c длинным длинным длинным длинным названием}}
\newcommand{\opponentTwoJobPlaceEn}
{\fixme{Place of work}}
%
\newcommand{\opponentTwoJobPost}       % Оппонент 2, должность
{\fixme{старший научный сотрудник}}
\newcommand{\opponentTwoJobPostEn}
{\fixme{position}}

%% \newcommand{\opponentThreeFio}         % Оппонент 3, ФИО
%% {\fixme{Фамилия Имя Отчество}}
%% \newcommand{\opponentThreeRegalia}     % Оппонент 3, регалии
%% {\fixme{кандидат физико-математических наук}}
%% \newcommand{\opponentThreeJobPlace}    % Оппонент 3, место работы
%% {\fixme{Основное место работы c длинным длинным длинным длинным названием}}
%% \newcommand{\opponentThreeJobPost}     % Оппонент 3, должность
%% {\fixme{старший научный сотрудник}}

\newcommand{\leadingOrganizationTitle} % Ведущая организация, дополнительные строки. Удалить, чтобы не отображать в автореферате
{\fixme{Федеральное государственное бюджетное образовательное учреждение высшего
профессионального образования с~длинным длинным длинным длинным названием}}

\newcommand{\defenseDate}              % Защита, дата
{\fixme{DD.MM.YYYY~г.~в~XX:00 часов}}
\newcommand{\defenseDateEn}
{\fixme{DD.MM.YYYY at XX:00}}
\newcommand{\defenseCouncilNumber}     % Защита, номер диссертационного совета
{11.21.00}
\newcommand{\defenseCouncilTitle}      % Защита, учреждение диссертационного совета
{Университета ИТМО}
\newcommand{\defenseCouncilAddress}    % Защита, адрес учреждение диссертационного совета
{Санкт-Петербург, Кронверкский пр.\,49, лит.\,А, ауд.~\fixme{XXX}}
\newcommand{\defenseCouncilPhone}      % Телефон для справок
{\fixme{+7~(0000)~00-00-00}}

\newcommand{\defenseSecretaryFio}      % Секретарь диссертационного совета, ФИО
{Андреев Юрий Сергеевич}
\newcommand{\defenseSecretaryFioEn}
{Andreev IUrii Sergeevich}
\newcommand{\defenseSecretaryRegalia}  % Секретарь диссертационного совета, регалии
{кандидат технических наук}  % Для сокращений есть ГОСТы, например: ГОСТ Р 7.0.12-2011 + http://base.garant.ru/179724/#block_30000
\newcommand{\defenseSecretaryRegaliaEn}
{Candidate of Technical Sciences}

\newcommand{\synopsisLibrary}          % Автореферат, название библиотеки
{Универcитета ИТМО  по адресу: 197101, Санкт-Петербург, Кронверкский~пр., д.\,49, лит.\,А и на сайте \mbox{\url{https://dissovet.itmo.ru}}}
\newcommand{\synopsisLibraryEn}
{49 Kronversky pr., \mbox{Saint-Petersburg}, Russia and on \url{https://dissovet.itmo.ru} website}
\newcommand{\synopsisDate}             % Автореферат, дата рассылки
{\fixme{DD mmmmmmmm}\the\year~года}

% To avoid conflict with beamer class use \providecommand
\providecommand{\keywords}%            % Ключевые слова для метаданных PDF диссертации и автореферата
{}
