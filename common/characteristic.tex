
{\actuality} Мировые тенденции в области промышленного производства свидетельствуют о том, что современные методы автоматизации и роботизации достигли определенного технологического барьера. Как показала практика, сейчас существенное увеличение производительности, а следовательно, и снижение себестоимости выпускаемой продукции возможно только при постоянном увеличении объемов производства.

Таким образом, в последние десятилетия все средства промышленной автоматизации и роботизации были направлены на массовое производство, способное обеспечить высокое качество выпускаемой продукции при минимальной себестоимости. Однако последние тенденции говорят о том, что массовое производство перестает оправдывать себя. Это связано с тем фактом, что подавляющее большинство современных изделий массового производства включают себя не только физические, но информационные компоненты. Повсеместное развитие средств информатизации и телекоммуникации повлекло за собой появление новых и существенную модернизацию старых изделий. В литературе такие изделия называются <<smart things>> или <<умные вещи>>. 

<<Умные вещи>> являются совокупностью физических и информационных компонентов. Информационные компоненты задают алгоритм работы <<умной вещи>>, позволяют ей осуществлять постоянную самодиагностику и возможность взаимодействия по сети.  Совокупность <<умных вещей>> образует так называемый <<Интернет вещей>>, включающий в себя не только распределенную сеть <<умных вещей>>, но единый облачный сервис для управления ими. Все эти понятия сведенные вместе стали основой новой концепции, именуемой <<кибер-физическая система>>.

Очевидно, что развитие информационных компонентов кибер-физических систем происходит гораздо быстрее, чем физических. Последнее обусловлено более коротким циклом производства программных продуктов в сравнении с физическими объектами, а также более низкой себестоимостью разработки программных продуктов. Вследствие этого возникло определенное технологическое отставание физических компонентов кибер-физических систем по сравнению с информационными.

Как показывает практика, единственным способом избавиться от этого отставания является ускорение темпов промышленного производства за счет более быстрой смены номенклатуры выпускаемых изделий, а также внедрение новых технологических процессов.

Постоянная смена номенклатуры и появление новых видов изделий требуют дополнительных научно-исследовательских и опытно-конструкторских работ, что привело к возникновению нового типа производственных компаний, именуемых малыми инновационными предприятиями (сокр. \textit{МИП}) или стартапами. Цель МИП – непрерывная модернизация существующих, а также проектирование и разработка новых образцов высокотехнологичной продукции в условиях мелкосерийного и единичного производства.

Для достижения поставленной цели МИП должны обладать парком разнообразного оборудования, позволяющего выполнять самые разнообразные технологические операции: от механической обработки изделий и создания электронных компонентов до автоматизированного контроля готовой продукции. Однако большинство МИП не в состоянии обеспечить себя всем необходимым оборудованием и поэтому вынуждены заниматься только разработкой конструкторской документации, передавая производство другим компаниям, обозначаемым термином ОЕМ~(от англ. \textit{Original Equipment Manufacturer}). Принимая во внимание все вышесказанное, необходимо отметить, что такой подход обладает рядом существенных недостатков.

Во-первых, OEM компании в основном заинтересованы в крупных заказах, т.\,к. для них смена номенклатуры связана с необходимостью каждый раз перестраивать производственный процесс. Соответственно, существенная часть времени уходит на технологическую подготовку производства, что значительно повышает себестоимость единицы продукции при работе с малыми партиями. Во-вторых, работа с OEM компаниями увеличивает время вывода на рынок новых видов продукции. Это может быть связано со множеством факторов таких, как необходимость заключения договора на производство и его юридического сопровождения, необходимость согласования конструкторской и технологической документации, передаваемой OEM компании, перегруженность производства OEM компании и т.\,д. В-третьих, риск потери интеллектуальной собственности, связанный с передачей полного комплекта документации на выпускаемые изделия. В-четвертых, OEM компании занимаются только производством по готовой документации, соответственно вопрос создания прототипов изделий, предсерийных образцов и установочных партий изделий для МИП остается нерешенным.

Отдельно стоит отметить, что в рамках Национальной технологической инициативы\footnote{Автономная некоммерческая организация <<Платформа Национальной технологической инициативы>> (сокр. \textit{АНО НТИ}) "--- некоммерческая организация созданная Постановлением председателя Правительства РФ Д.\,А.~Медведева. Разработка НТИ началась в соответствии с поручением Президента России В.\,В.~Путина по реализации послания Федеральному Собранию от 4 декабря 2014 года.} особое внимание уделяется развитию так называемых научно-технологических центров~(сокр. \textit{НТЦ}). Подобные организации реализуют замкнутого цикла исследований и производства. Для подобных организаций также необходимо наличие широкого парка различного технологического оборудования, ведь многие разработки, которые ведутся в НТЦ являются государственной или коммерческой тайной и применение подхода OEM, с передачей конструкторской или технологической документации сторонним фирмам (особенно зарубежным) просто недопустимо.  

Одним из способов решения вышеозначенных проблем является применение модульного технологического оборудования с числовым программным управлением (ЧПУ). Модульное оборудование с ЧПУ представляет собой совокупность независимых модулей, каждый из которых выполняет определенное действие (например, операцию обработки или контроля, перемещение и т.\,д.). На модули накладываются конструктивные параметрические ограничения. Модули объединены единой системой числового программного управления за счет использования стандартизированного и документированного протокола взаимодействия, позволяющего осуществлять децентрализованное взаимодействие как на уровне единицы модульного оборудования, так и на уровне производственной ячейки. 

Модульный принцип построения оборудования с числовым программным управлением освещен в работах таких авторов, как: Аверьянов О.\,И., Светик Дж. (Jozef Svetl\'ik), Йошими И. (Yoshimi Ito) и другие. Также образцы модульного оборудования с ЧПУ выпускаются некоторыми зарубежными и отечественными компаниями. Тем не менее, подавляющее большинство исследований направлены на рассмотрение только конструктивных особенностей и методов проектирования модульного технологического оборудования. Более того, почти все работы связаны исключительно с металлорежущими станками и не рассматривают другие виды обработки.

Также недостаточно проработаны аспекты проектирования и создания гибридного модульного оборудования, включающего в себя несколько видов обработки и/или контроля. Практически не рассматриваются вопросы повышение отказоустойчивости и ремонтопригодности, а также постепенной модернизации технологического оборудования за счёт использования модульного принципа.

Серийно выпускаемые модульные системы являются закрытыми как программно, так и аппаратно, то есть не позволяют в процессе эксплуатации создавать свои модули, равно как и изменять/дополнять программное обеспечение системы числового программного управления. 

Новые подходы к проектированию модульного оборудования требуют совершенствования системы числового программного управления, в частности разработки открытого программного интерфейса и единого протокола взаимодействия модулей, удовлетворяющего требованиям современных телекоммуникационных сетей.

Остается открытым вопрос унификации и стандартизации модульного оборудования. В частности, на сегодняшний день существует всего один действующий стандарт унификации изделий (ГОСТ~23945.0-80), а также несколько рекомендаций и руководящих документов\footnote{РД~50-632-87, Р~50-54-7-87, Р~50-54-102-88, Р~50-54-103-88.}, регламентирующих параметризацию и модульные конструкции. В июне 2017 года был выпущен первый международный стандарт из серии DIN--VDI/VDE/NAMUR 2658, на текущий момент включающий в себя уже семь разделов, посвященных применению модульных систем в промышленности. Однако стандарты данной серии являются достаточно общим и регламентируют модульную организацию всего производственного цикла как дискретных, так и непрерывных производств. Следовательно задача дальнейшего развития модульного подхода к проектированию технологического оборудования и совершенствования алгоритмического, программного и технического обеспечения систем числового программного управления таким оборудованием представляется актуальной.  

%{\progress} В последние пару десятилетий за рубежом активно проводятся исследования и разрабатываются технологии \ldots

{\aim} данной работы является разработка методики проектирования модульного технологического оборудования для единичного и мелкосерийного производства.

Для~достижения поставленной цели необходимо было решить следующие {\tasks}:
\begin{enumerate}[beginpenalty=10000] % https://tex.stackexchange.com/a/476052/104425
  \item Обосновать и сформировать принципы организации модульного оборудования.
  \item Разработать математическую модель параметрических рядов модулей и их ограничений.
  \item Разработать математическую модель оптимизации конфигураций модульного оборудования.
  \item Сформировать и обосновать способ оценки целесообразности применения модульного оборудования.
  \item Провести анализ существующих программных и аппаратных компонентов, пригодных для реализации модульного оборудования.
  \item Разработать архитектуру системы управления модульным оборудованием.
  \item Описать протокол межмодульного и внешнего взаимодействия модульного технологического оборудования.
  \item Разработать тестовый образец модульного оборудования и его системы управления.
\end{enumerate}


{\novelty}
\begin{enumerate}[beginpenalty=10000] % https://tex.stackexchange.com/a/476052/104425
  \item Впервые \ldots
  \item Впервые \ldots
  \item Было выполнено оригинальное исследование \ldots
\end{enumerate}

{\influence} \ldots

{\methods} \ldots

{\defpositions}
\begin{enumerate}[beginpenalty=10000] % https://tex.stackexchange.com/a/476052/104425
  \item Первое положение
  \item Второе положение
  \item Третье положение
  \item Четвертое положение
\end{enumerate}

{\reliability} полученных результатов определяется полнотой рассмотренного материала на достаточно высоком научно-теоретическом уровне. Все положения,  рассмотренные в диссертации, основательно проверены и научно обоснованны. Достигнутые результаты, изложенные в заключении диссертационной работы, соотносятся с поставленной целью и сформулированными задачами. Результаты проведённого исследования находятся в полном соответствии с результатами, полученными другими авторами, работающими в данной области исследований.


{\probation}
Основные результаты работы докладывались~на:
\begin{enumerate}[beginpenalty=10000]
	\item IEEE 15th International Conference on Industrial Informatics (INDIN-2017).
	\item IEEE 17th International Conference on Industrial Informatics (INDIN-2019).
	\item IEEE 1st International Conference on Industrial Cyber-Physical Systems (ICPS-2018).
	\item IEEE 3rd International Conference on Industrial Cyber-Physical Systems (ICPS-2020).
	\item 2017 IEEE 20th Conference of Open Innovations Association {FRUCT-20}.
	\item 2017 IEEE 21st Conference of Open Innovations Association {FRUCT-21}.
	\item 2018 IEEE 22nd Conference of Open Innovations Association {FRUCT-22}.
	\item 2018 IEEE 23rd Conference of Open Innovations Association {FRUCT-23}.
	\item 2019 IEEE 25th Conference of Open Innovations Association {FRUCT-25}.
	\item 2020 IEEE 26th Conference of Open Innovations Association {FRUCT-26}.
	\item 2020 IEEE 28th Conference of Open Innovations Association {FRUCT-28}.
	\item 2020 International Multi-Conference on Industrial Engineering and Modern Technologies (FarEastCon).
	\item The 1st International Conference on Computer Technology Innovations dedicated to the 100th anniversary of the Gorky House of Scientists of Russian Academy of Science (ICCTI-2020).
	\item VI Конгресс молодых учёных (2017).
	\item VII Конгресс молодых ученых (2018).
	\item VIII Конгресс молодых ученых (2019).
	\item IX Конгресс молодых ученых (2020).
	\item X Конгресс молодых ученых (2021).
	\item XLV научная и учебно-методическая конференция Университета \mbox{ИТМО} (2016).
	\item XLVI научная и учебно-методическая конференция Университета \mbox{ИТМО} (2017).
	\item XLVII научная и учебно-методическая конференция Университета \mbox{ИТМО} (2018).
	\item XLVIII научная и учебно-методическая конференция Университета \mbox{ИТМО} (2019).
	\item XLIX научная и учебно-методическая конференция Университета \mbox{ИТМО} (2020).
	\item XLX научная и учебно-методическая конференция Университета \mbox{ИТМО} (2021).
\end{enumerate}

{\contribution} Все результаты, представленные в диссертации, получены лично автором либо при его непосредственном участии. Автор принимал активное участие в разработке \dots Непосредственно автором предложена \dots

{\implementation} Результаты диссертационной работы использовались при проведении фундаментальных и прикладных научных исследований:

\begin{enumerate}[beginpenalty=10000]
	\item Научно-исследовательская работа, выполняемая в рамках Университета ИТМО на тему <<Разработка методов интеллектуального управления киберфизическими системами с использованием квантовых технологий>>  \textnumero 617026.
	\item Научно-исследовательская работа, выполняемая в рамках Университета ИТМО на тему <<Управление киберфизическими системами>>  \textnumero 718546.
	\item Научно-исследовательская работа, выполняемая в рамках Университета ИТМО на тему <<Разработка методов создания и внедрения киберфизических систем>>  \textnumero 619296.
	\item Научно-исследовательская работа, выполняемая в рамках Университета ИТМО на тему <<Методы искусственного интеллекта для киберфизических систем>>  \textnumero 620164.
\end{enumerate}


\ifnumequal{\value{bibliosel}}{0}
{%%% Встроенная реализация с загрузкой файла через движок bibtex8. (При желании, внутри можно использовать обычные ссылки, наподобие `\cite{vakbib1,vakbib2}`).
    {\publications} Основные результаты по теме диссертации изложены
    в~XX~печатных изданиях,
    X из которых изданы в журналах, рекомендованных ВАК,
    X "--- в тезисах докладов.
}%
{%%% Реализация пакетом biblatex через движок biber
    \begin{refsection}[bl-author, bl-registered]
        % Это refsection=1.
        % Процитированные здесь работы:
        %  * подсчитываются, для автоматического составления фразы "Основные результаты ..."
        %  * попадают в авторскую библиографию, при usefootcite==0 и стиле `\insertbiblioauthor` или `\insertbiblioauthorgrouped`
        %  * нумеруются там в зависимости от порядка команд `\printbibliography` в этом разделе.
        %  * при использовании `\insertbiblioauthorgrouped`, порядок команд `\printbibliography` в нём должен быть тем же (см. biblio/biblatex.tex)
        %
        % Невидимый библиографический список для подсчёта количества публикаций:
        \printbibliography[heading=nobibheading, section=1, env=countauthorvak,          keyword=biblioauthorvak]%
        \printbibliography[heading=nobibheading, section=1, env=countauthorwos,          keyword=biblioauthorwos]%
        \printbibliography[heading=nobibheading, section=1, env=countauthorscopus,       keyword=biblioauthorscopus]%
        \printbibliography[heading=nobibheading, section=1, env=countauthorconf,         keyword=biblioauthorconf]%
        \printbibliography[heading=nobibheading, section=1, env=countauthorother,        keyword=biblioauthorother]%
        \printbibliography[heading=nobibheading, section=1, env=countregistered,         keyword=biblioregistered]%
        \printbibliography[heading=nobibheading, section=1, env=countauthorpatent,       keyword=biblioauthorpatent]%
        \printbibliography[heading=nobibheading, section=1, env=countauthorprogram,      keyword=biblioauthorprogram]%
        \printbibliography[heading=nobibheading, section=1, env=countauthor,             keyword=biblioauthor]%
        \printbibliography[heading=nobibheading, section=1, env=countauthorvakscopuswos, filter=vakscopuswos]%
        \printbibliography[heading=nobibheading, section=1, env=countauthorscopuswos,    filter=scopuswos]%
        %
        \nocite{*}%
        %
        {\publications} Основные результаты по теме диссертации изложены в~\arabic{citeauthor}~печатных изданиях,
        \arabic{citeauthorvak} из которых изданы в журналах, рекомендованных ВАК\sloppy%
        \ifnum \value{citeauthorscopuswos}>0%
            , \arabic{citeauthorscopuswos} "--- в~периодических изданиях, индексируемых Web of~Science и Scopus\sloppy%
        \fi%
        \ifnum \value{citeauthorconf}>0%
            , \arabic{citeauthorconf} "--- в~тезисах докладов.
        \else%
            .
        \fi%
        \ifnum \value{citeregistered}=1%
            \ifnum \value{citeauthorpatent}=1%
                Зарегистрирован \arabic{citeauthorpatent} патент.
            \fi%
            \ifnum \value{citeauthorprogram}=1%
                Зарегистрирована \arabic{citeauthorprogram} программа для ЭВМ.
            \fi%
        \fi%
        \ifnum \value{citeregistered}>1%
            Зарегистрированы\ %
            \ifnum \value{citeauthorpatent}>0%
            \formbytotal{citeauthorpatent}{патент}{}{а}{}\sloppy%
            \ifnum \value{citeauthorprogram}=0 . \else \ и~\fi%
            \fi%
            \ifnum \value{citeauthorprogram}>0%
            \formbytotal{citeauthorprogram}{программ}{а}{ы}{} для ЭВМ.
            \fi%
        \fi%
        % К публикациям, в которых излагаются основные научные результаты диссертации на соискание учёной
        % степени, в рецензируемых изданиях приравниваются патенты на изобретения, патенты (свидетельства) на
        % полезную модель, патенты на промышленный образец, патенты на селекционные достижения, свидетельства
        % на программу для электронных вычислительных машин, базу данных, топологию интегральных микросхем,
        % зарегистрированные в установленном порядке.(в ред. Постановления Правительства РФ от 21.04.2016 N 335)
    \end{refsection}%
    \begin{refsection}[bl-author, bl-registered]
        % Это refsection=2.
        % Процитированные здесь работы:
        %  * попадают в авторскую библиографию, при usefootcite==0 и стиле `\insertbiblioauthorimportant`.
        %  * ни на что не влияют в противном случае
        %\nocite{vakbib2}%vak
        %\nocite{patbib1}%patent
        %\nocite{progbib1}%program
        %\nocite{bib1}%other
        %\nocite{confbib1}%conf
    \end{refsection}%
        %
        % Всё, что вне этих двух refsection, это refsection=0,
        %  * для диссертации - это нормальные ссылки, попадающие в обычную библиографию
        %  * для автореферата:
        %     * при usefootcite==0, ссылка корректно сработает только для источника из `external.bib`. Для своих работ --- напечатает "[0]" (и даже Warning не вылезет).
        %     * при usefootcite==1, ссылка сработает нормально. В авторской библиографии будут только процитированные в refsection=0 работы.
}