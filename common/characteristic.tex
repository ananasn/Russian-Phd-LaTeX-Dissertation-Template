
{\actuality} Обзор, введение в тему, обозначение места данной работы в
мировых исследованиях и~т.\:п.

{\aim} данной работы является \ldots

Для~достижения поставленной цели необходимо было решить следующие {\tasks}:
\begin{enumerate}[beginpenalty=10000] % https://tex.stackexchange.com/a/476052/104425
  \item Исследовать, разработать, вычислить и~т.\:д. и~т.\:п.
  \item Исследовать, разработать, вычислить и~т.\:д. и~т.\:п.
  \item Исследовать, разработать, вычислить и~т.\:д. и~т.\:п.
  \item Исследовать, разработать, вычислить и~т.\:д. и~т.\:п.
\end{enumerate}


{\novelty}
\begin{enumerate}[beginpenalty=10000] % https://tex.stackexchange.com/a/476052/104425
  \item Впервые \ldots
  \item Впервые \ldots
  \item Было выполнено оригинальное исследование \ldots
\end{enumerate}

{\influence} \ldots

{\methods} \ldots

{\defpositions}
\begin{enumerate}[beginpenalty=10000] % https://tex.stackexchange.com/a/476052/104425
  \item Первое положение
  \item Второе положение
  \item Третье положение
  \item Четвертое положение
\end{enumerate}
В папке Documents можно ознакомиться в решением совета из Томского ГУ
в~файле \verb+Def_positions.pdf+, где обоснованно даются рекомендации
по~формулировкам защищаемых положений.

{\reliability} \note[green]{полученных результатов определяется полнотой рассмотренного материала на достаточно высоком научно-теоретическом уровне. Все положения,  рассмотренные в диссертации, основательно проверены и научно обоснованны. Достигнутые результаты, изложенные в заключении диссертационной работы, соотносятся с поставленной целью и сформулированными задачами. Результаты проведённого исследования находятся в полном соответствии с результатами, полученными другими авторами, работающими в данной области исследований.}


{\probation}
\note[green]{Основные результаты работы докладывались~на:}
\begin{enumerate}[beginpenalty=10000]
	\item IEEE 15th International Conference on Industrial Informatics (INDIN-2017).
	\item IEEE 17th International Conference on Industrial Informatics (INDIN-2019).
	\item IEEE 1st International Conference on Industrial Cyber-Physical Systems (ICPS-2018).
	\item IEEE 3rd International Conference on Industrial Cyber-Physical Systems (ICPS-2020).
	\item 2017 IEEE 20th Conference of Open Innovations Association {FRUCT-20}.
	\item 2017 IEEE 21st Conference of Open Innovations Association {FRUCT-21}.
	\item 2018 IEEE 22nd Conference of Open Innovations Association {FRUCT-22}.
	\item 2018 IEEE 23rd Conference of Open Innovations Association {FRUCT-23}.
	\item 2019 IEEE 25th Conference of Open Innovations Association {FRUCT-25}.
	\item 2020 IEEE 26th Conference of Open Innovations Association {FRUCT-26}.
	\item 2020 IEEE 26th Conference of Open Innovations Association {FRUCT-28}.
	\item I Международная конференция по инновационным компьютерным технологиям, посвященная 100-летию Дома ученых им.\,М.\,Горького РАН (ICCTI-2020).
	\item IV Всероссийский конгресс молодых учёных (2015).
	\item V Всероссийский конгресс молодых учёных (2016).
	\item VI Конгресс молодых учёных (2017).
	\item VII Конгресс молодых ученых (2018).
	\item VIII Конгресс молодых ученых (2019).
	\item IX Конгресс молодых ученых (2020).
	\item XLV научная и учебно-методическая конференция Университета {ИТМО} (2016).
	\item XLVI научная и учебно-методическая конференция Университета {ИТМО} (2017).
	\item XLVII научная и учебно-методическая конференция Университета {ИТМО} (2018).
	\item XLVIII научная и учебно-методическая конференция Университета {ИТМО} (2019).
	\item XLIX научная и учебно-методическая конференция Университета {ИТМО} (2020).
\end{enumerate}

{\contribution} \note[green]{Все результаты, представленные в диссертации, получены лично автором либо при его непосредственном участии. Автор принимал активное участие в разработке \ldots. Непосредственно автором предложена  \ldots}

{\implementation} \note[green]{Результаты диссертационной работы использовались при проведении фундаментальных и прикладных научных исследований:}

\begin{enumerate}[beginpenalty=10000]
	\item Научно-исследовательская работа, выполняемая в рамках Университета ИТМО на тему <<Разработка методов интеллектуального управления киберфизическими системами с использованием квантовых технологий>>  \textnumero 617026.
	\item Научно-исследовательская работа, выполняемая в рамках Университета ИТМО на тему <<Управление киберфизическими системами>>  \textnumero 718546.
	\item Научно-исследовательская работа, выполняемая в рамках Университета ИТМО на тему <<Разработка методов создания и внедрения киберфизических систем>>  \textnumero 619296.
	\item Научно-исследовательская работа, выполняемая в рамках Университета ИТМО на тему <<Методы искусственного интеллекта для киберфизических систем>>  \textnumero 620164.
\end{enumerate}


\ifnumequal{\value{bibliosel}}{0}
{%%% Встроенная реализация с загрузкой файла через движок bibtex8. (При желании, внутри можно использовать обычные ссылки, наподобие `\cite{vakbib1,vakbib2}`).
    {\publications} Основные результаты по теме диссертации изложены
    в~XX~печатных изданиях,
    X из которых изданы в журналах, рекомендованных ВАК,
    X "--- в тезисах докладов.
}%
{%%% Реализация пакетом biblatex через движок biber
    \begin{refsection}[bl-author, bl-registered]
        % Это refsection=1.
        % Процитированные здесь работы:
        %  * подсчитываются, для автоматического составления фразы "Основные результаты ..."
        %  * попадают в авторскую библиографию, при usefootcite==0 и стиле `\insertbiblioauthor` или `\insertbiblioauthorgrouped`
        %  * нумеруются там в зависимости от порядка команд `\printbibliography` в этом разделе.
        %  * при использовании `\insertbiblioauthorgrouped`, порядок команд `\printbibliography` в нём должен быть тем же (см. biblio/biblatex.tex)
        %
        % Невидимый библиографический список для подсчёта количества публикаций:
        \printbibliography[heading=nobibheading, section=1, env=countauthorvak,          keyword=biblioauthorvak]%
        \printbibliography[heading=nobibheading, section=1, env=countauthorwos,          keyword=biblioauthorwos]%
        \printbibliography[heading=nobibheading, section=1, env=countauthorscopus,       keyword=biblioauthorscopus]%
        \printbibliography[heading=nobibheading, section=1, env=countauthorconf,         keyword=biblioauthorconf]%
        \printbibliography[heading=nobibheading, section=1, env=countauthorother,        keyword=biblioauthorother]%
        \printbibliography[heading=nobibheading, section=1, env=countregistered,         keyword=biblioregistered]%
        \printbibliography[heading=nobibheading, section=1, env=countauthorpatent,       keyword=biblioauthorpatent]%
        \printbibliography[heading=nobibheading, section=1, env=countauthorprogram,      keyword=biblioauthorprogram]%
        \printbibliography[heading=nobibheading, section=1, env=countauthor,             keyword=biblioauthor]%
        \printbibliography[heading=nobibheading, section=1, env=countauthorvakscopuswos, filter=vakscopuswos]%
        \printbibliography[heading=nobibheading, section=1, env=countauthorscopuswos,    filter=scopuswos]%
        %
        \nocite{*}%
        %
        {\publications} Основные результаты по теме диссертации изложены в~\arabic{citeauthor}~печатных изданиях,
        \arabic{citeauthorvak} из которых изданы в журналах, рекомендованных ВАК\sloppy%
        \ifnum \value{citeauthorscopuswos}>0%
            , \arabic{citeauthorscopuswos} "--- в~периодических научных журналах, индексируемых Web of~Science и Scopus\sloppy%
        \fi%
        \ifnum \value{citeauthorconf}>0%
            , \arabic{citeauthorconf} "--- в~тезисах докладов.
        \else%
            .
        \fi%
        \ifnum \value{citeregistered}=1%
            \ifnum \value{citeauthorpatent}=1%
                Зарегистрирован \arabic{citeauthorpatent} патент.
            \fi%
            \ifnum \value{citeauthorprogram}=1%
                Зарегистрирована \arabic{citeauthorprogram} программа для ЭВМ.
            \fi%
        \fi%
        \ifnum \value{citeregistered}>1%
            Зарегистрированы\ %
            \ifnum \value{citeauthorpatent}>0%
            \formbytotal{citeauthorpatent}{патент}{}{а}{}\sloppy%
            \ifnum \value{citeauthorprogram}=0 . \else \ и~\fi%
            \fi%
            \ifnum \value{citeauthorprogram}>0%
            \formbytotal{citeauthorprogram}{программ}{а}{ы}{} для ЭВМ.
            \fi%
        \fi%
        % К публикациям, в которых излагаются основные научные результаты диссертации на соискание учёной
        % степени, в рецензируемых изданиях приравниваются патенты на изобретения, патенты (свидетельства) на
        % полезную модель, патенты на промышленный образец, патенты на селекционные достижения, свидетельства
        % на программу для электронных вычислительных машин, базу данных, топологию интегральных микросхем,
        % зарегистрированные в установленном порядке.(в ред. Постановления Правительства РФ от 21.04.2016 N 335)
    \end{refsection}%
    \begin{refsection}[bl-author, bl-registered]
        % Это refsection=2.
        % Процитированные здесь работы:
        %  * попадают в авторскую библиографию, при usefootcite==0 и стиле `\insertbiblioauthorimportant`.
        %  * ни на что не влияют в противном случае
        \nocite{vakbib2}%vak
        \nocite{patbib1}%patent
        \nocite{progbib1}%program
        \nocite{bib1}%other
        \nocite{confbib1}%conf
    \end{refsection}%
        %
        % Всё, что вне этих двух refsection, это refsection=0,
        %  * для диссертации - это нормальные ссылки, попадающие в обычную библиографию
        %  * для автореферата:
        %     * при usefootcite==0, ссылка корректно сработает только для источника из `external.bib`. Для своих работ --- напечатает "[0]" (и даже Warning не вылезет).
        %     * при usefootcite==1, ссылка сработает нормально. В авторской библиографии будут только процитированные в refsection=0 работы.
}

При использовании пакета \verb!biblatex! будут подсчитаны все работы, добавленные
в файл \verb!biblio/author.bib!. Для правильного подсчёта работ в~различных
системах цитирования требуется использовать поля:
\begin{itemize}
        \item \texttt{authorvak} если публикация индексирована ВАК,
        \item \texttt{authorscopus} если публикация индексирована Scopus,
        \item \texttt{authorwos} если публикация индексирована Web of Science,
        \item \texttt{authorconf} для докладов конференций,
        \item \texttt{authorpatent} для патентов,
        \item \texttt{authorprogram} для зарегистрированных программ для ЭВМ,
        \item \texttt{authorother} для других публикаций.
\end{itemize}
Для подсчёта используются счётчики:
\begin{itemize}
        \item \texttt{citeauthorvak} для работ, индексируемых ВАК,
        \item \texttt{citeauthorscopus} для работ, индексируемых Scopus,
        \item \texttt{citeauthorwos} для работ, индексируемых Web of Science,
        \item \texttt{citeauthorvakscopuswos} для работ, индексируемых одной из трёх баз,
        \item \texttt{citeauthorscopuswos} для работ, индексируемых Scopus или Web of~Science,
        \item \texttt{citeauthorconf} для докладов на конференциях,
        \item \texttt{citeauthorother} для остальных работ,
        \item \texttt{citeauthorpatent} для патентов,
        \item \texttt{citeauthorprogram} для зарегистрированных программ для ЭВМ,
        \item \texttt{citeauthor} для суммарного количества работ.
\end{itemize}
% Счётчик \texttt{citeexternal} используется для подсчёта процитированных публикаций;
% \texttt{citeregistered} "--- для подсчёта суммарного количества патентов и программ для ЭВМ.

Для добавления в список публикаций автора работ, которые не были процитированы в
автореферате, требуется их~перечислить с использованием команды \verb!\nocite! в
\verb!Synopsis/content.tex!.
