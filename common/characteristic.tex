
{\actuality} Рассматривая тенденции развития современной промышленности, нельзя не отметить постепенный переход к новым формам организации взаимодействия компонентов производственного оборудования. На предприятиях все чаще создаются единые центры управления оборудованием, внедряются новые методы онлайн мониторинга и контроля за производственным процессом. Происходит плавное размытие границ между физическими компонентами (станками, промышленными роботами и т.\:д.) и виртуальной моделью производства. Появилось новое понятие: Индустриальная Кибер-Физическая Система~(ИКФС). ИКФС "--- это информационно-технологическая концепция, подразумевающая интеграцию вычислительных ресурсов в физические процессы. Очевидно, что основным связующим звеном в подобных системах являются вычислительные сети. К данному компоненту ИКФС предъявляются наиболее жесткие требования, т.\:к. от надежности и отказоустойчивости сети зависит работоспособность всей системы в целом. Необходимо отметить, что в настоящее время на смену интеграции существующих средств автоматизации в единую централизованную сеть, приходят гибкие распределенные производственные комплексы. Происходит постепенный переход от понятия <<интеграция>> к понятию <<интероперабельность>>. Однако до сих пор основными компонентами любого автоматизированного производства являются станки с числовым программным управлением и промышленные роботы. Любое подобное устройство само по себе является достаточно сложным, а объединение их в сеть с помощью SCADA-систем еще больше усложняет процесс управления подобным производством.

Монолитная архитектура управления станочным оборудованием и промышленными роботами ведет к необходимости создания монолитных SCADA-систем. Все это плохо сочетается с основным принципом ИКФС "--- упрощение процесса производства за счет создания динамичной распределенной среды, которая связывает множество интеллектуальных устройств, способных воспринимать окружающую среду и выполнять соответствующие действия автономно.  Таким образом, можно прийти к выводу, что для создания высокоэффективной ИКФС следует разработать новую распределенную архитектуру оборудования с ЧПУ.  Основа данной архитектуры "--- открытый интерфейс, позволяющий отдельным компонентам оборудования оставаться автономными, но при этом способными общаться с другими компонентами при необходимости. Очевидно, что создание такого интерфейса невозможно без унификации протокола передачи данных. Данный протокол должен работать поверх уже существующих сетевых протоколов и давать возможность прозрачного обмена данными как между процессами операционной системы, так и между узлами гетерогенный сети (причем в качестве узла может выступать как программный сервис, работающий под управлением операционной системы, так и физический контроллер управления).

Современные системы управления отличаются комплексностью и монолитностью. Каждый их элемент "--- это <<вещь в себе>> с жёсткой и иерархической архитектурой. Всё в подобных системах направлено на обеспечение качества, надежности и бесперебойной работы. Инертность подобных систем заставляет разработчиков АСУ ТП следовать этому же принципу монолитности, потому что оборудование с централизованным управлением нельзя эффективно внедрить в децентрализованную производственную среду.

Вследствие этого, упор делается на интеграцию, то есть на объединение разрозненных компонентов в единую производственную систему, а не на интероперабельность "--- создание открытого интерфейса взаимодействия, позволяющего отдельным компонентам оставаться автономными, но способными общаться с другими компонентами в случае необходимости. Соответственно, желательно переходить к разработке оборудования с модульной архитектурой.

Подобная архитектура базируется на двух основных постулатах: унификации и гибридизации. Под унификацией понимается открытая программно-аппаратная структура, позволяющая создавать новые типы оборудования и программного обеспечения по принципу <<интеллектуального конструктора>>. Унификация достигается за счет разбиения единого изделия на крупные взаимозаменяемые блоки с четким описанием входных и выходных параметров каждого блока.

Обобщая, можно отметить, что практически любое промышленное оборудование состоит из следующих основных частей:

\begin{enumerate}
	\item Рабочий орган.
	\item Координатное шасси, перемещающее рабочий орган в пространстве.
	\item Блок числового программного управления.
\end{enumerate}

Очевидно, что координатное шасси является наиболее универсальным блоком, на который могут быть установлены различные рабочие органы, за счет чего может быть изменен тип оборудования. Например, сменив фрезерную головку на лазерную, можно сделать из фрезерного станка гравировальный или форматно-раскроечный, а поставив экструдер для пластика "--- 3D-принтер.

Применение же принципа гибридизации позволяет создавать установки, являющиеся совокупностью уже существующих. Например, можно сконфигурировать аддитивно-субстрактивную машину для быстрого прототипирования, сочетающую в себе характеристики 3D-принтера и трёхкоординатного фрезерного станка или совместить фрезерную головку для черновой обработки заготовок с лазером для полирования определенных поверхностей.

Для этого на базе спецификации и ограничений рабочего органа могут быть созданы гибридные фрезерно-печатающие или фрезерно-полирующие головки  для послойного выращивания с последующей селективной доработкой некоторых поверхностей напечатанного изделия фрезером или лазером.

Однако не стоит забывать о том, что смена рабочего органа "--- это не только механическое изменение параметров оборудования, но и смена алгоритма управления. Очевидно, что блок числового программного управления должен заранее знать о каждом рабочем органе, который может быть установлен на шасси. Очевидно, что при таком подходе мы опять получим монолитную систему с иерархическим управлением.

Поэтому необходимо выделить базовую часть архитектуры, определить спецификацию протокола взаимодействия и создания новых программно-аппаратных модулей, которые могут быть динамически включены в систему. Очевидно, что к базовой части будет относиться алгоритм управления электрическими приводами координатного шасси, а все остальное будет являться внешними компонентами.

Для реализации подобной архитектуры логично создать распределенную сеть с диспетчером программно-аппаратных модулей. Все модули подключаются по сети, при этом физически подключение может быть как проводному, так и по беспроводному протоколу. Каждый модуль получает динамический IP-адрес по DHCP, затем отправляет широковещательный запрос в поисках диспетчера, диспетчер отвечает по заранее определенному протоколу, модуль регистрируется в реестре диспетчера сообщая свои основные физические параметры "--- какие команды он может получать, какие данные передает базовому контроллеру шасси.

Подобный подход позволяет полностью изменить стиль управления производственной средой, позволяя добиться бесшовного объединения различных производственных установок в единую сеть. Однако, не все так просто. Ведь система управление промышленным оборудованием "--- это не только алгоритм управления, но и база данных, пользовательский интерфейс и компоненты взаимодействия с физической средой (также именуемые встроенными системами или микроконтроллерами). Для разделения подобных целостных систем необходим системный подход, в качестве которого авторами предлагается использование архитектурного шаблона микросервисов.

Необходимо отметить, что упрощение архитектуры промышленного оборудования за счет разделения на унифицированные блоки влечет за собой неминуемое усложнение связей между компонентами оборудования, которые по определению должны находится в единой децентрализованной сети.  Однако на данный момент не существует единого стандарта промышленных распределённых сетей и организация взаимодействия элементов сети у разных компаний сделана по-своему.

Исходя из этого, одним из краеугольных камней существующих производственных кибер-физических систем является сетевая реализация, которая должна отвечать требованиям отказоустойчивости, защищённости и скорости работы. Чем больше производственного оборудования сосредоточено на предприятии/участке, тем сложнее организовать взаимодействие участников сети друг с другом, поскольку в стандартной сети типа <<звезда>> центральный узел вынужден работать с большим числом подключений. Кроме того, топология распределения устройств и датчиков в производственной зоне может существенно отличаться в зависимости от типа производственного процесса. Исходя из вышесказанного, разрабатываемая сетевая архитектура должна обладать способностью к самоорганизации и самовосстановлению.

Одним из вариантов реализации сетевой архитектуры может служить технология ячеистых сетей или mesh-сетей. Оборудование позволяет выбирать оптимальные частоты и маршруты передачи данных в автоматическом режиме. Стандартная промышленная сеть подразумевает, что каждое устройство является закрытой единицей оборудования, состоящей из взаимодействующих по сети модулей, управление которыми осуществляется специальным управляющим модулем (диспетчером) []. 

Очевидно, что для такой архитектуры больше подходит классическая архитектура, типа <<звезда>>, которая позволяет осуществлять быстрое подчиненное управление каждым из модулей без необходимости поиска его в сети и выстраивания оптимального маршрута. Таким образом, каждая единица для решения непосредственно задач производства обязана оставаться закрытой и монолитной "--- чем проще модель управления, тем она надежнее и отказоустойчивее. С другой стороны, кибер-физическая производственная система "--- это не просто совокупность промышленного оборудования и датчиков производственной среды. Второй важной сущностью ИКФС является полная модель производства, именуемая также <<цифровым двойником>>.

Именно наличие <<цифрового двойника>> является основным отличием ИКФС от обычных промышленных сенсорных сетей или сетей управления. Управляющее воздействие на модель должно приводить изменению физического компонента ИКФС и наоборот. При этом работа системы в целом должна оставаться абсолютно прозрачной для пользователя, например, замена физических компонентов ИКФС на виртуальные не должна влиять на работу системы в целом. Очевидно, что для реализации подобной концепции модель должна быть максимально подробной. Именно поэтому предлагается использовать именно не монолитный подход к проектированию промышленного оборудования, а модульный с возможностью объединения всех компонентов ИКФС в одну децентрализованную сеть. 

При этом получается вариация так называемого холонического подхода [], когда одна и та же система может быть как иерархической, так и гетерархической в зависимость от условий работы. Поясним это на примере проектируемой системы. Как уже было отмечено ранее каждая единица промышленного оборудования является совокупностью модулей, управляемых общим диспетчером по сети. При этом каждый модуль является автономной сущностью со своей внутренней логикой работы, актуаторам, датчиками и т.д. При децентрализованной архитектуре сети данные от каждого модуля поступают не только непосредственному диспетчеру, к которому подключен модуль, но и напрямую в модель ИКФС.

Нештатная ситуация или отказ любого из модулей оборудования будет сразу транслирована по сети, при этом <<цифровой двойник>>, основываясь на собственной логике, сможет сразу принять решение о возможности продолжения работы единицы оборудования, в модуле которого произошел сбой (например, сбой случился в неответсвенном или не задействованном на данный момент модуле), либо об оперативном изменении производственного процесса и передачей задачи на другую (работоспособную или незанятую ) единицу оборудования. В дополнение к этому данная информация будет распространена по всей сети (нельзя забывать, что ИКФС "--- это еще и интеграция с облачными сервисам), причем это будет работать даже если отказ произойдет в модуле диспетчера.

Очевидно, что при таком подходе наиболее разумно перейти на беспроводную организацию каналов передачи данных. В случае, когда топология сети будет постоянно изменяться, установка проводных соединений будет сложной и дорогой, к тому же на участке могут быть места с затруднённым доступом. Проводные соединения будут неизбежно повреждаться и перетираться в местах изгибов, что прямо влияет на общую надёжность сетевой инфраструктуры.

Таким образом, можно сформулировать следующие требования к ячеистой сети, используемой для интеграции модульного оборудования в ИКФС:

\begin{enumerate}
	\item Использование беспроводного способа передачи данных.
	\item Открытость и возможность интеграции с другими система поддержки производственного процесса: SCADA, облачные сервисы и т.\:д.
	\item Высокая отказоустойчивость и возможность к самовосстановлению.
	\item Встроенная система обеспечения безопасности передачи данных.
	\item Использование открытых стандартов.
	\item Сеть должна представлять собой цифровую модель производства, являясь основой <<цифрового двойника>>.
\end{enumerate}


{\progress} В последние пару десятилетий за рубежом активно проводятся исследования и разрабатываются технологии \ldots

{\aim} данной работы является \ldots

Для~достижения поставленной цели необходимо было решить следующие {\tasks}:
\begin{enumerate}[beginpenalty=10000] % https://tex.stackexchange.com/a/476052/104425
  \item Исследовать, разработать, вычислить и~т.\:д. и~т.\:п.
  \item Исследовать, разработать, вычислить и~т.\:д. и~т.\:п.
  \item Исследовать, разработать, вычислить и~т.\:д. и~т.\:п.
  \item Исследовать, разработать, вычислить и~т.\:д. и~т.\:п.
\end{enumerate}


{\novelty}
\begin{enumerate}[beginpenalty=10000] % https://tex.stackexchange.com/a/476052/104425
  \item Впервые \ldots
  \item Впервые \ldots
  \item Было выполнено оригинальное исследование \ldots
\end{enumerate}

{\influence} \ldots

{\methods} \ldots

{\defpositions}
\begin{enumerate}[beginpenalty=10000] % https://tex.stackexchange.com/a/476052/104425
  \item Первое положение
  \item Второе положение
  \item Третье положение
  \item Четвертое положение
\end{enumerate}

{\reliability} полученных результатов определяется полнотой рассмотренного материала на достаточно высоком научно-теоретическом уровне. Все положения,  рассмотренные в диссертации, основательно проверены и научно обоснованны. Достигнутые результаты, изложенные в заключении диссертационной работы, соотносятся с поставленной целью и сформулированными задачами. Результаты проведённого исследования находятся в полном соответствии с результатами, полученными другими авторами, работающими в данной области исследований.


{\probation}
Основные результаты работы докладывались~на:
\begin{enumerate}[beginpenalty=10000]
	\item IEEE 15th International Conference on Industrial Informatics (INDIN-2017).
	\item IEEE 17th International Conference on Industrial Informatics (INDIN-2019).
	\item IEEE 1st International Conference on Industrial Cyber-Physical Systems (ICPS-2018).
	\item IEEE 3rd International Conference on Industrial Cyber-Physical Systems (ICPS-2020).
	\item 2017 IEEE 20th Conference of Open Innovations Association {FRUCT-20}.
	\item 2017 IEEE 21st Conference of Open Innovations Association {FRUCT-21}.
	\item 2018 IEEE 22nd Conference of Open Innovations Association {FRUCT-22}.
	\item 2018 IEEE 23rd Conference of Open Innovations Association {FRUCT-23}.
	\item 2019 IEEE 25th Conference of Open Innovations Association {FRUCT-25}.
	\item 2020 IEEE 26th Conference of Open Innovations Association {FRUCT-26}.
	\item 2020 IEEE 28th Conference of Open Innovations Association {FRUCT-28}.
	\item 2020 International Multi-Conference on Industrial Engineering and Modern Technologies (FarEastCon).
	\item The 1st International Conference on Computer Technology Innovations dedicated to the 100th anniversary of the Gorky House of Scientists of Russian Academy of Science (ICCTI-2020).
	\item IV Всероссийский конгресс молодых учёных (2015).
	\item V Всероссийский конгресс молодых учёных (2016).
	\item VI Конгресс молодых учёных (2017).
	\item VII Конгресс молодых ученых (2018).
	\item VIII Конгресс молодых ученых (2019).
	\item IX Конгресс молодых ученых (2020).
	\item XLV научная и учебно-методическая конференция Университета \mbox{ИТМО} (2016).
	\item XLVI научная и учебно-методическая конференция Университета \mbox{ИТМО} (2017).
	\item XLVII научная и учебно-методическая конференция Университета \mbox{ИТМО} (2018).
	\item XLVIII научная и учебно-методическая конференция Университета \mbox{ИТМО} (2019).
	\item XLIX научная и учебно-методическая конференция Университета \mbox{ИТМО} (2020).
\end{enumerate}

{\contribution} Все результаты, представленные в диссертации, получены лично автором либо при его непосредственном участии. Автор принимал активное участие в разработке \dots Непосредственно автором предложена \dots

{\implementation} Результаты диссертационной работы использовались при проведении фундаментальных и прикладных научных исследований:

\begin{enumerate}[beginpenalty=10000]
	\item Научно-исследовательская работа, выполняемая в рамках Университета ИТМО на тему <<Разработка методов интеллектуального управления киберфизическими системами с использованием квантовых технологий>>  \textnumero 617026.
	\item Научно-исследовательская работа, выполняемая в рамках Университета ИТМО на тему <<Управление киберфизическими системами>>  \textnumero 718546.
	\item Научно-исследовательская работа, выполняемая в рамках Университета ИТМО на тему <<Разработка методов создания и внедрения киберфизических систем>>  \textnumero 619296.
	\item Научно-исследовательская работа, выполняемая в рамках Университета ИТМО на тему <<Методы искусственного интеллекта для киберфизических систем>>  \textnumero 620164.
\end{enumerate}


\ifnumequal{\value{bibliosel}}{0}
{%%% Встроенная реализация с загрузкой файла через движок bibtex8. (При желании, внутри можно использовать обычные ссылки, наподобие `\cite{vakbib1,vakbib2}`).
    {\publications} Основные результаты по теме диссертации изложены
    в~XX~печатных изданиях,
    X из которых изданы в журналах, рекомендованных ВАК,
    X "--- в тезисах докладов.
}%
{%%% Реализация пакетом biblatex через движок biber
    \begin{refsection}[bl-author, bl-registered]
        % Это refsection=1.
        % Процитированные здесь работы:
        %  * подсчитываются, для автоматического составления фразы "Основные результаты ..."
        %  * попадают в авторскую библиографию, при usefootcite==0 и стиле `\insertbiblioauthor` или `\insertbiblioauthorgrouped`
        %  * нумеруются там в зависимости от порядка команд `\printbibliography` в этом разделе.
        %  * при использовании `\insertbiblioauthorgrouped`, порядок команд `\printbibliography` в нём должен быть тем же (см. biblio/biblatex.tex)
        %
        % Невидимый библиографический список для подсчёта количества публикаций:
        \printbibliography[heading=nobibheading, section=1, env=countauthorvak,          keyword=biblioauthorvak]%
        \printbibliography[heading=nobibheading, section=1, env=countauthorwos,          keyword=biblioauthorwos]%
        \printbibliography[heading=nobibheading, section=1, env=countauthorscopus,       keyword=biblioauthorscopus]%
        \printbibliography[heading=nobibheading, section=1, env=countauthorconf,         keyword=biblioauthorconf]%
        \printbibliography[heading=nobibheading, section=1, env=countauthorother,        keyword=biblioauthorother]%
        \printbibliography[heading=nobibheading, section=1, env=countregistered,         keyword=biblioregistered]%
        \printbibliography[heading=nobibheading, section=1, env=countauthorpatent,       keyword=biblioauthorpatent]%
        \printbibliography[heading=nobibheading, section=1, env=countauthorprogram,      keyword=biblioauthorprogram]%
        \printbibliography[heading=nobibheading, section=1, env=countauthor,             keyword=biblioauthor]%
        \printbibliography[heading=nobibheading, section=1, env=countauthorvakscopuswos, filter=vakscopuswos]%
        \printbibliography[heading=nobibheading, section=1, env=countauthorscopuswos,    filter=scopuswos]%
        %
        \nocite{*}%
        %
        {\publications} Основные результаты по теме диссертации изложены в~\arabic{citeauthor}~печатных изданиях,
        \arabic{citeauthorvak} из которых изданы в журналах, рекомендованных ВАК\sloppy%
        \ifnum \value{citeauthorscopuswos}>0%
            , \arabic{citeauthorscopuswos} "--- в~периодических изданиях, индексируемых Web of~Science и Scopus\sloppy%
        \fi%
        \ifnum \value{citeauthorconf}>0%
            , \arabic{citeauthorconf} "--- в~тезисах докладов.
        \else%
            .
        \fi%
        \ifnum \value{citeregistered}=1%
            \ifnum \value{citeauthorpatent}=1%
                Зарегистрирован \arabic{citeauthorpatent} патент.
            \fi%
            \ifnum \value{citeauthorprogram}=1%
                Зарегистрирована \arabic{citeauthorprogram} программа для ЭВМ.
            \fi%
        \fi%
        \ifnum \value{citeregistered}>1%
            Зарегистрированы\ %
            \ifnum \value{citeauthorpatent}>0%
            \formbytotal{citeauthorpatent}{патент}{}{а}{}\sloppy%
            \ifnum \value{citeauthorprogram}=0 . \else \ и~\fi%
            \fi%
            \ifnum \value{citeauthorprogram}>0%
            \formbytotal{citeauthorprogram}{программ}{а}{ы}{} для ЭВМ.
            \fi%
        \fi%
        % К публикациям, в которых излагаются основные научные результаты диссертации на соискание учёной
        % степени, в рецензируемых изданиях приравниваются патенты на изобретения, патенты (свидетельства) на
        % полезную модель, патенты на промышленный образец, патенты на селекционные достижения, свидетельства
        % на программу для электронных вычислительных машин, базу данных, топологию интегральных микросхем,
        % зарегистрированные в установленном порядке.(в ред. Постановления Правительства РФ от 21.04.2016 N 335)
    \end{refsection}%
    \begin{refsection}[bl-author, bl-registered]
        % Это refsection=2.
        % Процитированные здесь работы:
        %  * попадают в авторскую библиографию, при usefootcite==0 и стиле `\insertbiblioauthorimportant`.
        %  * ни на что не влияют в противном случае
        \nocite{vakbib2}%vak
        \nocite{patbib1}%patent
        \nocite{progbib1}%program
        \nocite{bib1}%other
        \nocite{confbib1}%conf
    \end{refsection}%
        %
        % Всё, что вне этих двух refsection, это refsection=0,
        %  * для диссертации - это нормальные ссылки, попадающие в обычную библиографию
        %  * для автореферата:
        %     * при usefootcite==0, ссылка корректно сработает только для источника из `external.bib`. Для своих работ --- напечатает "[0]" (и даже Warning не вылезет).
        %     * при usefootcite==1, ссылка сработает нормально. В авторской библиографии будут только процитированные в refsection=0 работы.
}