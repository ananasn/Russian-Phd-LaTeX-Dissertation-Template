
{\actualityEn} Global trends in industrial production indicate that modern automation and robotization methods have reached a certain technological barrier. As practice has shown, now a significant increase in productivity, and, consequently, a decrease in the cost of production is possible only with a constant increase in production volumes.

Thus, in recent decades, all means of industrial automation and robotization have been aimed at mass production, capable of providing high quality products at the lowest cost. However, recent trends suggest that mass production is no longer justifying itself. This is due to the fact that the vast majority of modern mass-produced products include not only physical but informational components. The widespread development of information technology and telecommunications has led to the emergence of new and significant modernization of old products. In the literature, such products are called ``smart things'' or ``smart things''.

``Smart things'' are a collection of physical and informational components. Information components set the algorithm for the operation of the ``smart thing'', allow it to carry out constant self-diagnostics and the ability to interact over the network. The collection of ``smart things'' forms the so-called ``Internet of Things'', which includes not only a distributed network of ``smart things'', but a single cloud service for managing them. All these concepts brought together became the basis of a new concept called ``cyber-physical system''.

It is obvious that the development of information components of cyber-physical systems is much faster than physical ones. The latter is due to the shorter production cycle of software products in comparison with physical objects, as well as the lower cost of developing software products. As a result, a certain technological backwardness of the physical components of cyber-physical systems has arisen in comparison with the information ones.

As practice shows, the only way to get rid of this lag is to accelerate the pace of industrial production due to a faster change in the range of manufactured products, as well as the introduction of new technological processes.

The constant change of the nomenclature and the emergence of new types of products require additional research and development work, which has led to the emergence of a new type of manufacturing companies called small innovative enterprises (\textit {SIE}) or start-ups. The purpose of the SIE is the continuous modernization of existing ones, as well as the design and development of new samples of high-tech products in the conditions of small-scale and one-off production.

To achieve this goal, SIEs must have a fleet of various equipment that allows them to perform a wide variety of technological operations: from mechanical processing of products and the creation of electronic components to automated control of finished products. However, most of the SIEs are not able to provide themselves with all the necessary equipment and therefore are forced to deal only with the development of design documentation, transferring production to other companies, designated by the term OEM~(\textit {Original Equipment Manufacturer}). Taking into account all of the above, it should be noted that this approach has a number of significant disadvantages.

Firstly, OEM companies are mainly interested in large orders, i.\,e. for them, the change in the nomenclature is associated with the need to rebuild the production process each time. Accordingly, a significant part of the time is spent on technological preparation of production, which significantly increases the cost per unit of production when working with small batches. Secondly, working with OEM companies increases the time to market for new products. This may be due to many factors, such as the need to conclude a production contract and its legal support, the need to agree on design and technological documentation transferred to the OEM company, the overload of the production of the OEM company,~etc. Thirdly, the risk of loss of intellectual property associated with the transfer of a complete set of documentation for manufactured products. Fourthly, OEM companies are engaged only in manufacturing according to ready-made documentation, therefore, the issue of creating prototypes of products, pre-production samples and installation batches of products for SIE remains unresolved.

Separately, it should be noted that within the framework of the National Technological Initiative \footnote {Autonomous non-profit organization ``Platform of the National Technological Initiative'' is a non-profit organization created by the Decree of the Prime Minister of the Russian Federation D.\,A.~Medvedev. Development of National Technological Initiative began in accordance with the instruction of the President of Russia V.\,V.~Putin on the implementation of the message to the Federal Assembly from December 4, 2014.} special attention is paid to the development of the so-called scientific and technological centers. Such organizations implement a closed cycle of research and production. For such organizations, it is also necessary to have a wide fleet of various technological equipment, because many developments that are carried out in the R\&D Center are state or commercial secrets and the use of the OEM approach, with the transfer of design or technological documentation to third-party firms (especially foreign ones) is simply unacceptable.

One of the ways to solve the above problems is the use of modular technological equipment with numerical control (CNC). Modular CNC equipment is a collection of independent modules, each of which performs a specific action (for example, a processing or control operation, movement,~etc.). Modules are subject to design parametric constraints. The modules are united by a single numerical control system through the use of a standardized and documented interaction protocol, which allows decentralized interaction both at the level of a unit of modular equipment and at the level of a production cell.

The modular principle of constructing numerically controlled equipment is highlighted in the works of such authors as: Averyanov O.\,I., Jozef Svetl\'ik, Yoshimi Ito and others. Also, samples of modular CNC equipment are produced by some foreign and domestic companies. Nevertheless, the overwhelming majority of studies are aimed at considering only the design features and methods of designing modular technological equipment. Moreover, almost all work is related exclusively to metal cutting machines and does not consider other types of processing.

Also, the aspects of design and creation of hybrid modular equipment, including several types of processing and/or control, have not been sufficiently worked out. The issues of increasing fault tolerance and maintainability, as well as the gradual modernization of technological equipment through the use of the modular principle, are practically not considered.

Serially produced modular systems are closed both software and hardware, that is, they do not allow creating their own modules during operation, as well as changing/supplementing the software of the numerical control system.

New approaches to the design of modular equipment require the improvement of the numerical control system, in particular, the development of an open software interface and a unified protocol for the interaction of modules that meets the requirements of modern telecommunication networks.

The question of unification and standardization of modular equipment remains open. In particular, today there is only one current standard for product unification (GOST~23945.0-80), as well as several recommendations and guidelines\footnote{RD~50-632-87, R~50-54-7-87, R~50-54-102-88, R~50-54-103-88.}, regulating parameterization and modular designs. In June 2017, the first international standard from the DIN series was released---VDI/VDE/NAMUR 2658, which currently includes seven chapters devoted to the application of modular systems in industry. However, the standards of this series are quite general and regulate the modular organization of the entire production cycle, both discrete and continuous production. Consequently, the task of further developing a modular approach to the design of technological equipment and improving the algorithmic, software and technical support of numerical control systems for such equipment seems relevant.

%{\progressEn} In the last couple of decades, research and development of technologies have been actively carried out abroad. \ldots

{\aimEn} of this work is. \ldots

To~achieve this goal, it was necessary to solve the following {\tasksEn}:
\begin{enumerate}[beginpenalty=10000] % https://tex.stackexchange.com/a/476052/104425
  \item Research, develop, calculate,~etc.
  \item Research, develop, calculate,~etc.
  \item Research, develop, calculate,~etc.
  \item Research, develop, calculate,~etc.
\end{enumerate}


{\noveltyEn}
\begin{enumerate}[beginpenalty=10000] % https://tex.stackexchange.com/a/476052/104425
  \item For the first time \ldots
  \item ВFor the first timeпервые \ldots
  \item Original research was performed \ldots
\end{enumerate}

{\influenceEn} \ldots

{\methodsEn} \ldots

{\defpositionsEn}
\begin{enumerate}[beginpenalty=10000] % https://tex.stackexchange.com/a/476052/104425
	\item First position
	\item Second position
	\item Third position
	\item Fourth position
\end{enumerate}

{\reliabilityEn} of the results obtained is determined by the completeness of the material considered at a sufficiently high scientific and theoretical level. All the provisions considered in the dissertation are thoroughly verified and scientifically substantiated. The results achieved, stated in the conclusion of the dissertation work, correlate with the goal and the formulated tasks. The results of this study are in full agreement with the results obtained by other authors working in this area of research.


{\probationEn}
The main results of the work were reported at the following conferences:
\begin{enumerate}[beginpenalty=10000]
	\item IEEE 15th International Conference on Industrial Informatics (INDIN-2017).
	\item IEEE 17th International Conference on Industrial Informatics (INDIN-2019).
	\item IEEE 1st International Conference on Industrial Cyber-Physical Systems (ICPS-2018).
	\item IEEE 3rd International Conference on Industrial Cyber-Physical Systems (ICPS-2020).
	\item 2017 IEEE 20th Conference of Open Innovations Association {FRUCT-20}.
	\item 2017 IEEE 21st Conference of Open Innovations Association {FRUCT-21}.
	\item 2018 IEEE 22nd Conference of Open Innovations Association {FRUCT-22}.
	\item 2018 IEEE 23rd Conference of Open Innovations Association {FRUCT-23}.
	\item 2019 IEEE 25th Conference of Open Innovations Association {FRUCT-25}.
	\item 2020 IEEE 26th Conference of Open Innovations Association {FRUCT-26}.
	\item 2020 IEEE 28th Conference of Open Innovations Association {FRUCT-28}.
	\item 2020 International Multi-Conference on Industrial Engineering and Modern Technologies (FarEastCon).
	\item The 1st International Conference on Computer Technology Innovations dedicated to the 100th anniversary of the Gorky House of Scientists of Russian Academy of Science (ICCTI-2020).
	\item IV All-Russian Congress of Young Scientists (2015).
	\item V All-Russian Congress of Young Scientists (2016).
	\item VI Congress of Young Scientists (2017).
	\item VII Congress of Young Scientists (2018).
	\item VIII Congress of Young Scientists (2019).
	\item IX Congress of Young Scientists (2020).
	\item XLV scientific and educational-methodical conference of {ITMO} University (2016).
	\item XLVI scientific and educational-methodical conference of {ITMO} University (2017).
	\item XLVII scientific and educational-methodical conference of {ITMO} University (2018).
	\item XLVIII scientific and educational-methodical conference of {ITMO} University (2019).
	\item XLIX scientific and educational conference of {ITMO} University (2020).
\end{enumerate}

{\contributionEn} All the results presented in the dissertation were obtained by the author personally or with his direct participation. The author took an active part in the development of \dots Directly suggested by the author \dots

{\implementationEn} The results of the dissertation were used in fundamental and applied scientific research:

\begin{enumerate}[beginpenalty=10000]
	\item Research work carried out within the framework of ITMO University on the topic ``Development of methods for intelligent control of cyber-physical systems using quantum technologies'' \# 617026.
	\item Research work carried out within the framework of ITMO University on the topic ``Cyber-physical systems management''  \# 718546.
	\item Research work carried out within the framework of ITMO University on the topic ``Development of methods for creating and implementing cyber-physical systems'' \# 619296.
	\item Research work carried out within the framework of ITMO University on the topic ``Artificial Intelligence Methods for Cyber-Physical Systems \# 620164.
\end{enumerate}


\ifnumequal{\value{bibliosel}}{0}
{%%% Встроенная реализация с загрузкой файла через движок bibtex8. (При желании, внутри можно использовать обычные ссылки, наподобие `\cite{vakbib1,vakbib2}`).
    {\publications} Основные результаты по теме диссертации изложены
    в~XX~печатных изданиях,
    X из которых изданы в журналах, рекомендованных ВАК,
    X "--- в тезисах докладов.
}%
{%%% Реализация пакетом biblatex через движок biber
    \begin{refsection}[bl-author, bl-registered]
        % Это refsection=1.
        % Процитированные здесь работы:
        %  * подсчитываются, для автоматического составления фразы "Основные результаты ..."
        %  * попадают в авторскую библиографию, при usefootcite==0 и стиле `\insertbiblioauthor` или `\insertbiblioauthorgrouped`
        %  * нумеруются там в зависимости от порядка команд `\printbibliography` в этом разделе.
        %  * при использовании `\insertbiblioauthorgrouped`, порядок команд `\printbibliography` в нём должен быть тем же (см. biblio/biblatex.tex)
        %
        % Невидимый библиографический список для подсчёта количества публикаций:
        \printbibliography[heading=nobibheading, section=1, env=countauthorvak,          keyword=biblioauthorvak]%
        \printbibliography[heading=nobibheading, section=1, env=countauthorwos,          keyword=biblioauthorwos]%
        \printbibliography[heading=nobibheading, section=1, env=countauthorscopus,       keyword=biblioauthorscopus]%
        \printbibliography[heading=nobibheading, section=1, env=countauthorconf,         keyword=biblioauthorconf]%
        \printbibliography[heading=nobibheading, section=1, env=countauthorother,        keyword=biblioauthorother]%
        \printbibliography[heading=nobibheading, section=1, env=countregistered,         keyword=biblioregistered]%
        \printbibliography[heading=nobibheading, section=1, env=countauthorpatent,       keyword=biblioauthorpatent]%
        \printbibliography[heading=nobibheading, section=1, env=countauthorprogram,      keyword=biblioauthorprogram]%
        \printbibliography[heading=nobibheading, section=1, env=countauthor,             keyword=biblioauthor]%
        \printbibliography[heading=nobibheading, section=1, env=countauthorvakscopuswos, filter=vakscopuswos]%
        \printbibliography[heading=nobibheading, section=1, env=countauthorscopuswos,    filter=scopuswos]%
        %
        \nocite{*}%
        %
        {\publicationsEn} Основные результаты по теме диссертации изложены в~\arabic{citeauthor}~печатных изданиях,
        \arabic{citeauthorvak} из которых изданы в журналах, рекомендованных ВАК\sloppy%
        \ifnum \value{citeauthorscopuswos}>0%
            , \arabic{citeauthorscopuswos} "--- в~периодических научных журналах, индексируемых Web of~Science и Scopus\sloppy%
        \fi%
        \ifnum \value{citeauthorconf}>0%
            , \arabic{citeauthorconf} "--- в~тезисах докладов.
        \else%
            .
        \fi%
        \ifnum \value{citeregistered}=1%
            \ifnum \value{citeauthorpatent}=1%
                Зарегистрирован \arabic{citeauthorpatent} патент.
            \fi%
            \ifnum \value{citeauthorprogram}=1%
                Зарегистрирована \arabic{citeauthorprogram} программа для ЭВМ.
            \fi%
        \fi%
        \ifnum \value{citeregistered}>1%
            Зарегистрированы\ %
            \ifnum \value{citeauthorpatent}>0%
            \formbytotal{citeauthorpatent}{патент}{}{а}{}\sloppy%
            \ifnum \value{citeauthorprogram}=0 . \else \ и~\fi%
            \fi%
            \ifnum \value{citeauthorprogram}>0%
            \formbytotal{citeauthorprogram}{программ}{а}{ы}{} для ЭВМ.
            \fi%
        \fi%
        % К публикациям, в которых излагаются основные научные результаты диссертации на соискание учёной
        % степени, в рецензируемых изданиях приравниваются патенты на изобретения, патенты (свидетельства) на
        % полезную модель, патенты на промышленный образец, патенты на селекционные достижения, свидетельства
        % на программу для электронных вычислительных машин, базу данных, топологию интегральных микросхем,
        % зарегистрированные в установленном порядке.(в ред. Постановления Правительства РФ от 21.04.2016 N 335)
    \end{refsection}%
    \begin{refsection}[bl-author, bl-registered]
        % Это refsection=2.
        % Процитированные здесь работы:
        %  * попадают в авторскую библиографию, при usefootcite==0 и стиле `\insertbiblioauthorimportant`.
        %  * ни на что не влияют в противном случае
        %\nocite{vakbib2}%vak
        %\nocite{patbib1}%patent
        %\nocite{progbib1}%program
        %\nocite{bib1}%other
        %\nocite{confbib1}%conf
    \end{refsection}%
        %
        % Всё, что вне этих двух refsection, это refsection=0,
        %  * для диссертации - это нормальные ссылки, попадающие в обычную библиографию
        %  * для автореферата:
        %     * при usefootcite==0, ссылка корректно сработает только для источника из `external.bib`. Для своих работ --- напечатает "[0]" (и даже Warning не вылезет).
        %     * при usefootcite==1, ссылка сработает нормально. В авторской библиографии будут только процитированные в refsection=0 работы.
}