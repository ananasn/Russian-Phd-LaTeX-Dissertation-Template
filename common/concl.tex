%% Согласно ГОСТ Р 7.0.11-2011:
%% 5.3.3 В заключении диссертации излагают итоги выполненного исследования, рекомендации, перспективы дальнейшей разработки темы.
%% 9.2.3 В заключении автореферата диссертации излагают итоги данного исследования, рекомендации и перспективы дальнейшей разработки темы.

Значимость полученных в диссертационной работе результатов подчеркивается возможностью интеграции разработанных методик, подходов и алгоритмов в систему автоматизированного проектирования модульного технологического оборудования. Предложенные рекомендации по созданию узлов и агрегатов, а также блоков управления модульным технологическим оборудованием упростят его разработку и применение в условиях единичного и мелкосерийного производства.  В ходе исследования получены следующие основные теоретические и практические результаты:

\begin{enumerate}
  \item Разработано алгоритмическое и программно-техническое обеспечение модульного блока управления, включающее в себя набор программно-аппаратных средств для реализации децентрализованного сетевого взаимодействия модулей в проводной и беспроводной среде передачи данных и программное обеспечение для автоматической реконфигурации единицы модульного оборудования, позволяющее снизить трудоёмкость переналадки.  
  \item Разработана методика унификации модулей с электромагнитным креплением, включающая в себя способ определения параметров унификации и их ограничений и способ формирования параметрического ряда на основании сформулированных ограничений.
  \item Предложен критерий целесообразности применения модульного оборудования, основанный на анализе групповых технологических процессов, позволяющий оценить  перспективы использования модульного оборудования для типовых технологических процессов, используемых на предприятии.
  \item Разработана методика оптимизации комплекта модульного оборудования, включающая в себя способ расчёта весовых коэффициентов целевой функции оптимизации и алгоритм многокритериальной оптимизации, основанный дискретно-событийном методе.
  \item На основании методики оптимизации комплекта модульного оборудования разработано программное обеспечение и проведен численный эксперимент, показавший повышение производительности производства группы изделий на~\SI{18}{\percent}.
  \item Для тестирования предложенной конструкции модульного оборудования, а также методик и алгоритмов работы с ним создан прототип модульной технологической платформы.
\end{enumerate}

Таким образом, на основании полученных результатов, цель данного диссертационного исследования по разработке модульного технологического оборудования для условий единичного и мелкосерийного производства можно считать достигнутой.

Полученные результаты соответствуют пункту 6 <<Разработка, исследование и внедрение новых видов технологического оборудования для изготовления деталей, сборки, регулировки, контроля и испытаний приборов>> паспорта специальности 05.11.14 "--- <<Технология приборостроения>>.
