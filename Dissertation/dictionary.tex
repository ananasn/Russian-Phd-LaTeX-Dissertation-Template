\chapter*{Словарь терминов}             % Заголовок
\addcontentsline{toc}{chapter}{Словарь терминов}  % Добавляем его в оглавление

%\textbf{TeX} : Cистема компьютерной вёрстки, разработанная американским профессором информатики Дональдом Кнутом

%\textbf{панграмма} : Короткий текст, использующий все или почти все буквы алфавита

\textbf{Экспонирование} : облучение актиничным излучением (излучение, которое оказывает фотографическое действие на светочувствительный материал) светочувствительного слоя через фотошаблон или с помощью управляемого луча.

\textbf{Фоторезист} : светочувствительный материал, изменяющий свои свойства (прежде всего "--- растворимость) под воздействием актиничного излучения. 

\textbf{Ламинирование} : наложение полимера на подложку с последующей прикаткой и охлаждением для получения армированных многослойных изделий, состоящих из нескольких слоев разнородных материалов.

\textbf{Проявление} : обработка экспонированной пленки фоторезиста с целью удаления необлученных участков для создания рельефного изображения

\textbf{Экструзия} : получение изделий из пластмасс и резиновых смесей в экструдере (машина для пластикации материалов и придания им формы путем продавливания через экструзионную головку)

\textbf{Экструзия заготовок} : Формообразование заготовок непрерывным или периодическим выдавливанием пластического материала через канал формующего инструмента "--- головку 

\textbf{Пайка готовым припоем} : способ пайки, при котором используется заранее приготовленный припой. В качестве припоя может использоваться металлический (полностью расплавляемый) или композиционный припой. В композиционном припое помимо металлической основы содержится тугоплавкий наполнитель (порошки, волокна, сетки), который сам не плавится, а при плавлении металла припоя образует разветвленную сеть капилляров, удерживающих под действием капиллярных сил его жидкую часть в зазоре между соединяемыми деталями (ГОСТ 17325-79).
