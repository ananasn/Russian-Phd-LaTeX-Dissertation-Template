\chapter{Аналитический обзор современного состояния исследований в предметной области}\label{ch:ch1}

\section{Определения рамок обзора}\label{sec:ch1/sec1}

\section{Модульные системы управления технологическим оборудованием}\label{sec:ch1/sec2}

Анализируя развитие современной промышленности, можно прийти к выводу о том, что на сегодняшний день наиболее перспективным направлением развития в этой области является создание гибких распределенных автоматизированных производственных линий. Четвертая промышленная революция и постепенное внедрение кибер-физических производственных систем по-новому определяют само понятие массового производства.

Происходит последовательный переход от <<жестких>> конвейерных решений к малым партиям, выполняемым по индивидуальным заказам. Также не останавливается развитие  концепции малых инновационных предприятий и стартапов. Все это приводит к изменению подхода к проектированию современного интеллектуального оборудования. 

Первые станки с ЧПУ появились еще в 50-х годах прошлого века, и с тех пор их развитие не прекращается. Тем не менее, вектор этого развития всегда был ориентирован исключительно на массовое производство. Системы с ЧПУ всегда были и остаются сложными, высокопроизводительными, а самое главное "--- очень дорогим. Процесс внедрения подобных систем очень долог, а время жизни современных автоматизированных производств может исчисляться десятилетиями.

Конечно, все это усложняет внедрение современных коммуникационных технологий. Любое изменение требует либо полного перестроения устоявшейся производственной системы, либо создания дополнительного слоя управления, который позволит связать устаревшее оборудование с современной кибер-физической системой.

Такой подход представляется наиболее целесообразным в переходный период, но в дальнейшем от него, безусловно, придется отказаться. Необходим пересмотр самой парадигмы проектирования оборудования с ЧПУ. Нужно рассматривать любое новое оборудование с точки зрения возможности включения его в единую информационно-телекоммуникационную среду с использование открытого протокола.

Попытки реализации подобного подхода делались уже неоднократно. Например, в работе Grigoriev and Martinov предлагается подход к построению переносимого ядра ЧПУ на основе платформы независимых библиотек. Открытая архитектура данной системы ЧПУ включает в себя уровни абстракции для реализации различных человеко-машинных интерфейсов, а также имеет возможность описания компонентов системы на различных языках программирования. Компоненты системы связываются между собой по шине Fieldbus
%~\cite{grigoriev2014research}
.
Bin et al. описывают открытую платформу для создания систем ЧПУ. Данная система состоит из набора универсальных компонентов, которые могут быть использованы повторно, а также коммуникационных модулей для их связи%
%\cite{bin2004research}

.

Morales-Velazquez et al. предлагают платформу с открытой архитектурой на основе многоагентной системы программно-аппаратных компонентов, именуемой MADCON. The Аппаратные блоки предлагаемой системы объединяют функции управления и мониторинга, обеспечивая открытую архитектуру на базе FPGA для реконфигурируемых приложений. Компоненты программного обеспечения используют структуру XML для файлов описания системы, собирая такие функции, как язык описания блок-схем и графический интерфейс пользовател/я%~\cite{morales2010open} 

В работах Verba et. al
%~\cite{verba2017platform}
и  Prazeres and Serrano
%~\cite{prazeres2016soft}
рассматривается очень интересная концепция Fog of Things. Данная концепция является развитие концепции Internet of Things, являющейся основой многих кибер-физических производственных систем. Fog of Things позволяет создать более однородную информационно-телекоммуникационную среду за счет совершенствования и упрощения протокола взаимодействия компонентов.

В работе [morales2010open] представлена система ЧПУ с открытой архитектурой, основанная на авторской мультиагентной аппаратно-программной платформе Multi-Agent Distributed CONtroller (MADCON). Эта система имеет возможность реконфигурируемости и адаптивности. При проектировании интеллектуальных приводов для этой системы были использован структурированный подход к созданию программных и аппаратных составляющих. Аппаратные блоки предлагаемой системы объединяют в себе функции управления и мониторинга, обеспечивая открытую архитектуру на основе ПЛИС для обеспечения реконфигурируемости. С другой стороны, программные компоненты разработаны с использованием языка XML для описания системы, что позволило отказаться от привычного программирования и создавать логику программных компонентов с помощью визуальных блок-схем. MADCON был применен в качестве системы управления модернизированным токарным станком с ЧПУ для управления и контроля с целью проверки предлагаемой архитектура, которая может быть применена для создания интеллектуального производства нового поколения системы.

В [han2007development] рассматривается контроллер с открытой архитектурой, который может стать основой для различного оборудования с ЧПУ. Разрабатываемый контроллер имеет модульную арихтектуру, а в качестве аппаратной платформы использует персональный компьютер. Основная цель работы "--- разработать методику построения программно-аппаратная платформа системы ЧПУ. Также исследуются методы статического моделирования контроллера с открытой архитектурой, включающие технологию объектно-ориентированного программирования, технологию применения динамических библиотеки и разделения системных модулей. Авторы обсуждают динамическое моделирование поведения и представление потока данных контроллера с открытой архитектурой, которые описаны с помощью иерархической модели конечного автомата. Для разработки библиотеки программных функциональных компонентов авторы создали модель многократно используемого программного модуля. В качестве испытательного стенд выступает 3-осевой фрезерный станок. Для данного станка был успешно разработан программный код системы ЧПУ, основанной на описанной библиотеке функциональных модулей с применением методики конфигурирования системы. Результаты экспериментов показывали, что, помимо увеличения степени повторного использования программного кода и открытости, применение вышеупомянутой методологии приводит к значительному сокращению времени разработки, а также стоимости обслуживания конечного оборудования с ЧПУ.


\section{Название раздела}\label{sec:ch1/sec3}

\subsection{Название подраздела}\label{subsec:ch1/sec3/sub1}

\subsection{Название подраздела}\label{subsec:ch1/sec3/sub2}

\subsection{Название подраздела}\label{subsec:ch1/sec3/sub3}

\subsection{Название подраздела}\label{subsec:ch1/sec3/sub4}

\section{Выводы по главе 1}

\FloatBarrier
