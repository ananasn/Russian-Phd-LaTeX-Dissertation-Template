\chapter{Методика информационного взаимодействия компонентов модульного технологического оборудования}\label{ch:ch3}

\section{Подсистема управления модульным оборудованием}\label{sec:ch3/sec1}

Современные системы управления отличаются комплексностью и монолитностью. Каждый их элемент "--- это <<вещь в себе>> с жёсткой и иерархической архитектурой. Всё в подобных системах направлено на обеспечение качества, надежности и бесперебойной работы. Инертность подобных систем заставляет разработчиков АСУ ТП следовать этому же принципу монолитности, потому что оборудование с централизованным управлением нельзя эффективно внедрить в децентрализованную производственную среду.

Вследствие этого, упор делается на интеграцию, то есть на объединение разрозненных компонентов в единую производственную систему, а не на интероперабельность – создание открытого интерфейса взаимодействия, позволяющего отдельным компонентам оставаться автономными, но способными общаться с другими компонентами в случае необходимости. Соответственно, желательно переходить к разработке оборудования с модульной архитектурой.

Подобная архитектура базируется на двух основных постулатах: унификации и гибридизации. Под унификацией понимается открытая программно-аппаратная структура, позволяющая создавать новые типы оборудования и программного обеспечения по принципу <<интеллектуального конструктора>>. Унификация достигается за счет разбиения единого изделия на крупные взаимозаменяемые блоки с четким описанием входных и выходных параметров каждого блока.

Обобщая, можно отметить, что практически любое промышленное оборудование состоит из следующих основных частей:

\begin{enumerate}
	\item Рабочий орган.
	\item Координатное шасси, перемещающее рабочий орган в пространстве.
	\item Блок числового программного управления.
\end{enumerate}

Очевидно, что координатное шасси является наиболее универсальным блоком, на который могут быть установлены различные рабочие органы, за счет чего может быть изменен тип оборудования. Например, сменив фрезерную головку на лазерную, можно сделать из фрезерного станка гравировальный или форматно-раскроечный, а поставив экструдер для пластика (filament) "--- 3D-принтер.

Применение же принципа гибридизации позволяет создавать установки, являющиеся совокупностью уже существующих. Например, можно сконфигурировать аддитивно-субстрактивную машину для быстрого прототипирования, сочетающую в себе характеристики 3D-принтера и трёхкоординатного фрезерного станка или совместить фрезерную головку для черновой обработки заготовок с лазером для полирования определенных поверхностей.

Для этого на базе спецификации и ограничений рабочего органа могут быть созданы гибридные фрезерно-печатающие или фрезерно-полирующие головки  для послойного выращивания с последующей селективной доработкой некоторых поверхностей напечатанного изделия фрезером или лазером.

Однако не стоит забывать о том, что смена рабочего органа "--- это не только механическое изменение параметров оборудования, но и смена алгоритма управления. Очевидно, что блок числового программного управления должен заранее знать о каждом рабочем органе, который может быть установлен на шасси. Очевидно, что при таком подходе мы опять получим монолитную систему с иерархическим управлением.

Поэтому необходимо выделить базовую часть архитектуры, определить спецификацию протокола взаимодействия и создания новых программно-аппаратных модулей, которые могут быть динамически включены в систему. Очевидно, что к базовой части будет относиться алгоритм управления электрическими приводами координатного шасси, а все остальное будет являться внешними компонентами.

Для реализации подобной архитектуры логично создать распределенную сеть с диспетчером программно-аппаратных модулей. Все модули подключаются по сети, при этом физически подключение может быть как проводному, так и по беспроводному протоколу. Каждый модуль получает динамический IP-адрес по DHCP, затем отправляет широковещательный запрос в поисках диспетчера, диспетчер отвечает по заранее определенному протоколу, модуль регистрируется в реестре диспетчера сообщая свои основные физические параметры "--- какие команды он может получать, какие данные передает базовому контроллеру шасси.

Подобный подход позволяет полностью изменить стиль управления производственной средой, позволяя добиться бесшовного объединения различных производственных установок в единую сеть. Однако, не все так просто. Ведь система управление промышленным оборудованием – это не только алгоритм управления, но и база данных, пользовательский интерфейс и компоненты взаимодействия с физической средой (также именуемые встроенными системами или микроконтроллерами). Для разделения подобных целостных систем необходим системный подход, в качестве которого авторами предлагается использование архитектурного шаблона микросервисов.


\section{Название раздела}\label{sec:ch3/sec2}

\section{Название раздела}\label{sec:ch3/sec3}

\subsection{Название подраздела}\label{subsec:ch3/sec3/sub1}

\subsection{Название подраздела}\label{subsec:ch3/sec3/sub2}

\subsection{Название подраздела}\label{subsec:ch3/sec3/sub3}

\subsection{Название подраздела}\label{subsec:ch3/sec3/sub4}

\FloatBarrier