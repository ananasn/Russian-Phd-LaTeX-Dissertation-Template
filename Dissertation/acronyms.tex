\chapter*{Список сокращений и условных обозначений} % Заголовок
\addcontentsline{toc}{chapter}{Список сокращений и условных обозначений}  % Добавляем его в оглавление
% при наличии уравнений в левой колонке значение параметра leftmargin приходится подбирать вручную

\begin{description}[align=right,leftmargin=3.5cm]

\item[ЧПУ] числовое программное управление
\item[ПО] программное обеспечение
\item[МИП] малое инновационное предприятие
\item[КИМ] контрольно-измерительная машина 
\item[АСУ ТП] автоматизированная система управления технологическими процессами
\item[АС] агрегатный станок
\item[НТИ] национальная технологическая инициатива
\item[НТЦ] научно-технологических центров
\item[ПЛК] программируемый логический контроллер
\item[ИКФС ]ндустриальная кибер-физическая система

\item[RMS] Reconfigurable Machine Tools
\item[OEM] Original Equipment Manufacturer
\item[FDM] Fused Deposition Modeling

\end{description}
