\chapter{Методика проектирования модульного технологического оборудования}\label{ch:ch2}

\section{Принципы унификации оборудования}

Проблема унификации, статус которой четко не определен, поскольку область распространения унификации очень обширна и включает продукцию, нормы и требования, процессы, методы и документы, "--- занимает особое место среди всех проблем стандартизации. Однако потребность в решении проблемы унификации по-прежнему существует, что подтверждается, в частности, действием таких нормативных документов, как ГОСТ В 20.39.105-83, ГОСТ РВ 15.207-2005, ГОСТ РВ 15.209 и ОТТ МО 1.1.5-84 (Часть 1), в которых установлены номенклатура и содержание основных требований к унификации, состав и содержание работ по унификации. На основе всесторонних исследований этой проблемы могут быть выбраны правильные решения конкретных задач в области унификации общепромышленной и обоснованы направления дальнейшего развития унификации в нашей стране.

На данном этапе унификация, как основополагающая составляющая процесса создания изделий, незаслуженно <<подзабыта>>, особенно в промышленном секторе экономики. Причиной этому являются такие факторы, как отсутствие совершенного нормативно-правового обеспечения, научнотехнических публикаций и учебных пособий в области унификации. Так, за последние 30 лет наиболее значимым для профессиональной деятельности в области унификации является лишь учебное пособие.

Причиной <<забытости>> унификации являются и существенные недочеты по управлению предприятиями на современном этапе вследствие остаточного влияния рыночной экономики в период 90-х-2005-х гг., и непонимание в решении более важных по ценности конечного результата задач с откладыванием их на неопределенный срок. Последнее особенно проявляется в ситуации, сложившейся сейчас в области унификации в промышленном секторе.

Все это привело к тому, что применение унификации как эффективного метода создания высококачественной продукции большинством современных руководителей не используется. Некоторые даже и не слышали о таком. Причинами пренебрежения принципами и методами унификации в промышленном секторе экономики являются также:

\begin{itemize}
	\item отсутствие экономической заинтересованности во внедрении на изделиях унифицированных составных частей, разработанных другими предприятиями;
	\item отсутствие практики конкурсных разработок важных составных частей и изделий;
	\item отсутствие четкой информация по унифицированным и стандартизованным составным частям и ряд других.
\end{itemize}

Известно также мнение: если ты что-то позаимствовал, то ты "--- не творец, "--- потому что десятилетиями в умах инженеров воспитывалось вполне обоснованное стремление к созданию только нового, оригинального.

В данной связи, необходима скорейшая переориентация унификации с решения текущих задач на перспективные, хотя и более сложные и трудоемкие, но более значимые и проблемные. Так, чтобы сделать современный промышленный сектор экономики высоко рациональным, управляемым в своем развитии, единым для производства оборонной и общепромышленной продукции, в его основу должны быть заложены:

\begin{enumerate}
	\item единство конструкторской и технологической подготовки производства;
	\item единая элементная база;
	\item легко перестраиваемое оборудование;
	\item широкая унификация и типизация.
\end{enumerate}
 

Построение такого сектора экономики возможно на основе модульного машиностроения, сущность которого заключается в создании элементной базы на модульном уровне, унификации модулей и компоновке из них изделий, оборудования и оснастки. Такой путь открывает перспективу резкого скачка в уменьшении многообразия изделий (систем) без ущерба для их функций при одновременном сбережении огромных объемов материальных, энергетических и интеллектуальных ресурсов.

Расширение сфер применения и усложнение конструкций изделий неизбежно приводит к использованию в конструкторской практике таких методов и приемов, которые позволяют увеличивать в разумных пределах разнообразие технических решений как на параметрическом уровне (построение рациональных рядов параметров и типоразмеров, которые формируются с учетом многообразия потребностей в изделиях), так и на уровне конструктивных исполнений изделий и их составных частей. Одним из эффективных путей в этом отношении является переход к разработке конструкций, основанный на широком использовании агрегатов, модулей и других элементов, выделенных из их состава в качестве новых самостоятельных изделий.

Использование агрегатов и модулей является основной предпосылкой для освоения таких форм конструкторской деятельности как агрегатное и модульное конструирование, рассматриваемых как один из эффективных способов конструирования в общей системе унификации изделий. Эти формы конструкторской деятельности осуществляются в соответствии с принципом целесообразной преемственности конструирования, который имеет большое значение для повышения эффективности конструкторских работ.

Принцип целесообразной преемственности заключается в том, что в конкретных условиях развития техники и производства существуют наиболее целесообразные в техническом или экономическом отношении пропорции изменяемых и повторяемых признаков исполнений (параметров и компонентов) изделий. При этом состав признаков исполнений должен быть таким, чтобы любое исполнение этого изделия могло реализовать весь комплекс его основных функций. В этом процессе, направленном на техническое перевооружение промышленности и повышение качества продукции, важнейшую роль играют унификация и типизация. Это утверждение вытекает из возможности того, что унификация позволяет обеспечить решение проблемы совместимости образцов, взаимозаменяемости их составными частями и комплектующими изделиями и создание оптимальных рядов изделий.

В свою очередь, типизация является процессом подведения многообразия деталей, процессов под какой-либо рациональный (в идеале "--- оптимальный тип), сведение множества к определенному числу типов конструкций машин, сооружений, технологических процессов, что:

\begin{enumerate}
	\item во-первых, существенно сокращает количество типоразмеров элементов изделий с неоспоримым преимуществом "--- резким удешевлением единицы продукции;
	\item во-вторых, при ограниченном числе типоразмеров легче обеспечить и их высокую надежность и технологичность;
	\item в-третьих, открывается возможность для упрощения процессов проектирования и изготовления изделий.
\end{enumerate}

Унификация присуща всей Вселенной в любой её точке. Вселенная, Солнечная система, Планетарная схема, Царства природы, Человек, Атом имеют, по некоторым критериям, аналогичные построение и проявления. Так, Природа нашей Планеты предлагает каждому живому существу определенные внешние условия и целый набор химических элементов, из которых живая клетка как конструктор выстраивает свое тело, в соответствии с программами, заложенными в ДНК. При этом всего четыре химических соединения (аденин, тимин, гуанин, цитозин) различным образом сочетаясь, порождают все многообразие генетической информации.

<<Природа, "--- как писал известный архитектор Г. В. Борисовский, "--- это как бы огромный <<космический завод>>, непрерывно поставляющий изделия <<массового, серийного производства>> "--- людей, кур, муравьев, бабочек, березы, сосны и т.\:д.>> <<Массовое производство, осуществляемое природой, представляет собой глубочайшую из научных истин>>, "--- утверждал английский физик Джордж Томпсон. <<Принцип массового производства>> он относит к законам, лежащим в основе строения Вселенной.

Все это говорит о том, что идея унификации возникла не на пустом месте. Ее подсказала сама Природа, которая, например, унифицирует конструкции, строя их из элементов одной и той же формы: лепестки цветов, семена злаков, головка чеснока, ягоды малины, чешуйки рыб, змей, шишек, панцири и т. д. Унификация в промышленной сфере имеет длительную историю со своими взлетами и падениями. С развитием массового производства и усложнением техники роль унификации становится одной из основных в решении задачи создания надежной и приемлемой в цене техники.

Не секрет, что современная техника является сложной и разнообразной. Эта техника характеризуется высокой динамичностью, обусловленной большим ростом потребностей в высококачественных изделиях, непрерывным расширением сфер их использования и быстрым моральным старением. Эта особенность развития техники накладывает отпечаток и на процесс ее проектирования, т.\:е. на конструкторскую деятельность. Чтобы шагать в ногу со временем, конструктор заранее должен предвидеть все невозможное многообразие ожидаемых условий использования разрабатываемых изделий и осознавать необходимость создания стратегии обновления базовых изделий и их исполнений в обозримой перспективе. В данном процессе для конструктора не последнюю роль наравне со стандартизацией играют унификация и типизация.

Унификация показана как система весов с переменной точкой опоры. <<Весы унификации>> предназначены для взвешивания потребностей, выраженных в желаемом техническом уровне разрабатываемой продукции с возможностями разработчика, выраженных в ограниченном объеме ресурсов (времени, финансах, материальных и человеческих ресурсах и т. п.). Перемещая точку опоры Уун (уровень унификации) к нулевой отметке мы повышаем технический уровень. Приближая уровень унификации к единице, мы экономим ресурсы.

Принципы разработки и внедрения в промышленность научно-методических основ унификации в 60-80-е годы рассматривались на высоком государственном уровне. В данном направлении ЦК КПСС и Совет Министров СССР в эти годы приняли ряд мер. Так, в технических заданиях на создание новой техники предусматривалось установление уровнями унификации и экспертиза проектов на соответствие заданным в технических заданиях уровням, На главных конструкторов была возложена персональная ответственность за достижение высоких уровней унификации разрабатываемых изделий. С 1972 г. введено было планирование уровней унификации для изделий и их групп, выполнение которого устанавливалось на государственном уровне в нормативных документах вида <<Руководящий документ>>. Последним из этих документов был РД 50-173-80.

\section{Математическая модель унификации, параметризации и оптимизации компонентов модульной технологической платформы}

\section{Модель оптимизации модульного оборудования при единичном и мелкосерийном производствах}

Рассмотрим основные понятия оптимизации по номенклатуре выпускаемых изделий. Пусть $N$ "--- номенклатура групп изделий, предлагаемая для выпуска, полученная на этапе маркетинговых исследований:

\[
N = {D_1, D_2, \ldots, D_i}, N = (D_i)_{i \in I},
\]

\noindent где $D_i$ "--- группа изделий. Группирование происходит по технологическому признаку [Митрофанов, стр. 43] "--- \textit{вид обработки}.

\[
D = {P_1, P_2, \ldots, P_j},
\]

\noindent где $P_j$ "--- групповая технологическая операция. В соответствие каждой группе ставится вероятность спроса:

\[
f: D_i \rightarrow \upsilon_i, D_i \in N, \upsilon_i \in V,
\]

\noindent где $V$ "--- множество вероятностей такое, что:

\[
\forall\upsilon\in V: \upsilon\leq 1 \wedge\upsilon > 0
\]

Определим композицию для нахождения наиболее вероятной группы или нескольких групп:

\[
g: V \rightarrow N', \forall\upsilon = V_max: \upsilon \in f(N'), V_max \in V, N' \subset N
f \circ g = h,
\]

\noindent где $N' = (D_j)_{j \in J}$ "--- результат композиции (семейство наиболее вероятных групп), $\bigcup N' = {p:p_1\in D_1 \vee p_2\in D_2 \vee\ldots\vee p_j\in D_j}$ "--- множество групповых операций в этих группах.

Фактическая номенклатура задаётся нечётким множеством:

\[
N'' = {(D, \upsilon(D)) \mid D \in N}
\]

Определить коэффициенты целесообразности использования модульности можно как:

\[
\epsilon = \frac{\big|\bigcup N' \big|}{|G|},
\]

\noindent где $G$ "--- домен операций, $\bigcup\limits_{i} N$. Если $\epsilon\rightarrow 0$ "--- модульность целесообразна, если $\epsilon\rightarrow 1$ "--- модульность нецелесообразна.

\section{Определение критерия оптимизации конфигурации модульного оборудования}

\section{Разработка параметрического ряда ширины присоединительной поверхности}

\section{Выводы по главе 2}


\FloatBarrier