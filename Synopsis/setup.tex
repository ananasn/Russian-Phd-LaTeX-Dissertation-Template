%%%%%%%%%%%%%%%%%%%%%%%%%%%%%%%%%%%%%%%%%%%%%%%%%%%%%%%
%%%% Файл упрощённых настроек шаблона автореферата %%%%
%%%%%%%%%%%%%%%%%%%%%%%%%%%%%%%%%%%%%%%%%%%%%%%%%%%%%%%

%%% Инициализирование переменных, не трогать!  %%%
\newcounter{showperssign}
\newcounter{showsecrsign}
\newcounter{showopplead}
%%%%%%%%%%%%%%%%%%%%%%%%%%%%%%%%%%%%%%%%%%%%%%%%%%%%%%%

%%% Список публикаций %%%
\makeatletter
\@ifundefined{c@usefootcite}{
  \newcounter{usefootcite}
  \setcounter{usefootcite}{0} % 0 --- два списка литературы;
                              % 1 --- список публикаций автора + цитирование
                              %       других работ в сносках
}{}
\makeatother

\makeatletter
\@ifundefined{c@bibgrouped}{
  \newcounter{bibgrouped}
  \setcounter{bibgrouped}{1}  % 0 --- единый список работ автора;
                              % 1 --- сгруппированные работы автора
}{}
\makeatother

%%% Область упрощённого управления оформлением %%%

%% Управление зазором между подрисуночной подписью и основным текстом %%
\setlength{\belowcaptionskip}{10pt plus 20pt minus 2pt}


%% Подпись таблиц %%

% смещение строк подписи после первой
%\newcommand{\tabindent}{0cm}

% тип форматирования таблицы
% plain --- название и текст в одной строке
% split --- название и текст в разных строках
%\newcommand{\tabformat}{plain}

%%% настройки форматирования таблицы `plain'

% выравнивание по центру подписи, состоящей из одной строки
% true  --- выравнивать
% false --- не выравнивать
%\newcommand{\tabsinglecenter}{false}

% выравнивание подписи таблиц
% justified   --- выравнивать как обычный текст
% centering   --- выравнивать по центру
% centerlast  --- выравнивать по центру только последнюю строку
% centerfirst --- выравнивать по центру только первую строку
% raggedleft  --- выравнивать по правому краю
% raggedright --- выравнивать по левому краю
%\newcommand{\tabjust}{justified}

% Разделитель записи «Таблица #» и названия таблицы
%\newcommand{\tablabelsep}{~\cyrdash\ }

%%% настройки форматирования таблицы `split'

% положение названия таблицы
% \centering   --- выравнивать по центру
% \raggedleft  --- выравнивать по правому краю
% \raggedright --- выравнивать по левому краю
%\newcommand{\splitformatlabel}{\raggedleft}

% положение текста подписи
% \centering   --- выравнивать по центру
% \raggedleft  --- выравнивать по правому краю
% \raggedright --- выравнивать по левому краю
%\newcommand{\splitformattext}{\raggedright}

%% Подпись рисунков %%
%Разделитель записи «Рисунок #» и названия рисунка
%\newcommand{\fisglabelsep}{~\cyrdash\ }  % (ГОСТ 2.105, 4.3.1)
                                        % "--- здесь не работает

%Демонстрация подписи диссертанта на автореферате
\setcounter{showperssign}{1}  % 0 --- не показывать;
                              % 1 --- показывать
%Демонстрация подписи учёного секретаря на автореферате
\setcounter{showsecrsign}{1}  % 0 --- не показывать;
                              % 1 --- показывать
%Демонстрация информации об оппонентах и ведущей организации на автореферате
\setcounter{showopplead}{1}   % 0 --- не показывать;
                              % 1 --- показывать

%%% Цвета гиперссылок %%%
% Latex color definitions: http://latexcolor.com/
\definecolor{linkcolor}{rgb}{0.9,0,0}
\definecolor{citecolor}{rgb}{0,0.6,0}
\definecolor{urlcolor}{rgb}{0,0,1}
%\definecolor{linkcolor}{rgb}{0,0,0} %black
%\definecolor{citecolor}{rgb}{0,0,0} %black
%\definecolor{urlcolor}{rgb}{0,0,0} %black
